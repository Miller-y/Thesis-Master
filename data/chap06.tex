% !Mode:: "TeX:UTF-8"
\chapter{协同系统结构和实验结果分析}
\label{chap:experiment_system_analysis}

\section{引言 (Introduction)}

现有的 Wi-Fi CSI 室内定位研究面临两大核心挑战:一是深度学习模型对大规模、高精度标记数据的依赖与低效的人工采集方式之间的矛盾;二是单一射频(RF)模态在复杂室内环境(如非视距 NLOS、强多径效应)下的鲁棒性不足。

针对上述瓶颈,本文提出了引入多线阵 CCD 光学定位系统的解决方案,旨在构建一种\textbf{“离线监督-在线互补”的分层协同架构}。该架构通过\textbf{“高精度数据基准构建”}与\textbf{“光电异构多模态互补”}两个维度,实现了从数据生产到系统应用的全流程优化。

% =========================================================================
% 6.2 节:将原“系统逻辑设计”与“硬件实现”整合,逻辑更连贯
% =========================================================================
\section{协同定位系统的实现架构}
\label{sec:system_implementation}

针对 Wi-Fi CSI 室内定位中存在的标记数据获取困难与非视距(NLOS)鲁棒性差这两大核心问题,本文首先从逻辑层面设计了“离线监督-在线互补”的协同架构,并基于该架构搭建了包含高精度光学捕捉与分布式射频感知的实验原型系统。

\subsection{总体逻辑架构设计}
\label{subsec:logic_architecture}
% [原 6.2 内容移动至此]

本架构利用多线阵 CCD 光学系统与 CSI 射频系统的异构特性,分别在数据构建阶段和实时定位阶段实现优势互补。

\begin{figure}[htbp]
    \centering
    % \includegraphics[width=0.8\textwidth]{figure_placeholder.png}
    \caption{光电协同室内定位系统整体架构图}
    \label{fig:system_framework}
\end{figure}

\subsubsection{数据层协同:高精度时空基准的自动化构建}
\label{subsubsec:data_level_synergy}

深度学习驱动的 CSI 定位方法的性能上限,本质上受制于训练数据的规模与标记质量。传统的“走-停-采(Stop-and-Go)”离散采集模式不仅耗时巨大,且人工定点引入的随机位置误差会造成显著的标签噪声(Label Noise),严重制约了模型对精细信号特征的学习。

为此,本系统引入多线阵 CCD 作为**“离线监督者”**,构建了自动化的流式数据生产管线:

\begin{enumerate}
    \item \textbf{连续流式采集(Continuous Streaming Acquisition)}:利用 CCD 系统对携带特定波长 LED 的移动目标进行毫米级精度的实时追踪,替代传统的人工静态打点。这种方法将离散的指纹采集升级为连续的轨迹记录,不仅将数据生产效率提升了数量级,更消除了人为操作带来的不确定性。
    \item \textbf{动态特征捕捉}:本文提出的多尺度时序卷积(MSTC)网络高度依赖于信号的时序连续性。CCD 系统提供的连续高频坐标流,使得模型能够有效捕捉目标运动状态下 CSI 信号的多普勒频移(Doppler Shift)与时变衰落特性,为引入平滑约束(Smoothness Loss)提供了物理层面的数据支撑。
\end{enumerate}

为了将异构传感器的原始数据转化为可供深度学习模型训练的标准数据集,本文实施了严格的时空对齐与预处理策略:

\begin{enumerate}
    \item \textbf{空间重采样与轨迹标准化}:针对原始轨迹数据的非均匀性,采用空间等间距采样策略。从 CCD 记录的连续轨迹中,以 $10\text{cm}$ 为步长提取锚点,在实验空间内总计构建了 48,000 个高精度标记点。
    \item \textbf{时域滑窗与特征匹配}:基于光电系统的同步时间戳,以每个空间锚点为中心,截取时间窗口长度不少于 100 帧的 CSI 数据序列。
    \item \textbf{标签插值}:利用插值理论对采样间隙进行平滑处理,确保每一帧射频特征均能映射到唯一的精确空间坐标,从而最大限度地降低标签噪声,构建出“射频-空间”精确映射的指纹库。
\end{enumerate}

\subsubsection{策略层协同:基于视距感知的光电异构互补}
\label{subsubsec:strategy_synergy}

除了作为离线训练阶段的真值生成器外,CCD 光学定位与 CSI 射频定位在物理特性上呈现出显著的**“正交互补性(Orthogonality)”**。本研究构建了基于视距(LOS)状态感知的自适应切换机制,以应对复杂的室内环境挑战:

\begin{itemize}
    \item \textbf{光学主导模式(Optical-Dominant Mode)}:在开阔的 LOS 场景下,光学信号未受遮挡。此时系统优先采用 CCD 输出,提供毫米级的绝对定位精度,同时利用该数据对 CSI 模型进行在线校准。
   \item \textbf{射频备份模式(RF-Backup Mode)}:当目标移动至障碍物后方或发生视觉遮挡(NLOS 场景)时,CCD 系统会因光路中断或方差剧增而失效。此时,系统自动切换至 CSI 定位模块。利用 2.4GHz Wi-Fi 信号优异的绕射与穿透能力(波长约 12.5cm),通过多径反射特征维持分米级的定位服务,确保位置感知的连续性。
\end{itemize}

这种“光电双模”架构有效规避了单一光学系统的视觉盲区和单一射频系统的环境敏感性,显著提升了系统在动态不可控环境下的可用性(Availability)与鲁棒性。

\subsection{光学监测节点:YOPS 硬件系统}
\label{subsec:ccd_hardware}
% [原 6.3.1 内容移动至此]

光学监测节点作为系统的“离线监督者”,主要负责高精度的三维坐标真值生成。该子系统(YOPS)采用了模块化的嵌入式设计,核心架构如图 6.1 所示。

\subsubsection{信号采集与处理单元}
考虑到系统对光斑中心提取的实时性要求,主控单元(MCU)选用了基于 Cortex\texttrademark-M4 内核的 **STM32F405** 高性能处理器。该芯片具备 168 MHz 的主频与多通道 DMA 控制器,能够满足高频线阵数据流的实时吞吐需求。

光学传感前端采用了 **TCD1304 线阵 CCD**。该传感器具有 $8\,\mu\text{m} \times 20\,\mu\text{m}$ 的像元尺寸与 3648 个有效像素,能够在保证高信噪比(High SNR)的同时提供微米级的光学分辨率。为了实现与上位机的高速数据交互,系统集成了 USB 2.0 高速物理层接口,确保了 $1\,\text{kHz}$ 以上的采样帧率下数据传输零丢包。

图6.2为本文所设计YOPS实物,其中MCU和CCD所在印制电路板(Printed Circuit Board, PCB)通过排针安装于线阵相机所在PCB背面。

\subsubsection{嵌入式控制逻辑}
软件架构旨在最小化从光子入射到坐标解算的系统延迟。不同于通用的分层架构,本系统采用了**中断驱动(Interrupt-Driven)**与**直接内存访问(DMA)**相结合的零拷贝传输机制:
\begin{itemize}
    \item \textbf{并行采集}:利用 MCU 的定时器阵列同步触发三个线阵 CCD 的积分时序,确保多视角数据的时间一致性。
    \item \textbf{流式传输}:采集到的模拟电压信号经 12-bit ADC 量化后,直接通过 USB 批量传输(Bulk Transfer)模式上传至主机,由上位机(基于 Qt/C++ 开发)完成光斑质心的高斯拟合与三维重建。
\end{itemize}

如图6.3所示。本文所设计嵌入式软件分为5层,其中硬件层为MCU底层代码,直接控制MCU中的寄存器;抽象层将硬件层代码进行封装,提供MCU硬件(定时器、ADC、SPI等)控制接口;设备层使用抽象层提供的硬件接口驱动线阵CCD、CC1101等外部设备;接口层对设备层代码进一步封装,为每个外设定义初始化、读写、卸载等接口;应用层在接口层的基础上根据需求控制设备执行相应任务,并完成与上位机和标记点控制系统的通信。

上位机使用C++编写,图形化界面基于Qt 5.15实现,用于读取并处理YOPS中传感器原始数据、运行本文设计YOPS标定方法完成传感器标定、光学定位算法、可视化传感器数据和测量结果。图6.4展示了上位机主界面,包括YOPS测量位姿3D显示窗口、CCD原始数据显示窗口等。通过主界面,可以打开YOPS标定页面执行标定任务。

\subsection{分布式射频感知网络:CSI 采集节点}
\label{subsec:csi_system}
% [原 6.3.2 内容移动至此]

与依赖专用时钟同步线缆的传统 MIMO 系统不同,本研究构建了基于商用现货(COTS)设备的分布式感知网络。该设计利用 **ESP32-S3** SoC 的低成本与高集成度特性,实现了感知节点在三维空间中的灵活部署。

\subsubsection{感知节点拓扑与空间覆盖}
为了构建全向覆盖的射频感知场,系统在 $2.5\text{m} \times 2.5\text{m} \times 2.5\text{m}$ 的实验空间内采用**“中心激励-边缘接收”**的拓扑结构,如图 \ref{fig:system_layout} 所示:
\begin{itemize}
    \item \textbf{激励源(Tx)}:1 个发射节点置于底面几何中心,持续广播全向探测包。
    \item \textbf{感知阵列(Rx)}:8 个接收节点均匀分布于空间四周侧面(高度 $0.7\text{m}$,间距 $1\text{m}$),形成对目标区域的包围式感知。
\end{itemize}

该布局确保了目标在区域内任意位置移动时,至少会切割多条射频链路的有效菲涅尔区(Fresnel Zone),从而引发可观测的多径扰动特征。

\subsubsection{异构网络同步与数据汇聚}
由于分布式节点缺乏统一的物理时钟源,本系统设计了基于 **SNTP(Simple Network Time Protocol)** 的软同步机制以实现多模态对齐:

\begin{enumerate}
    \item \textbf{网络对齐}:所有 ESP32 节点与上位机处于同一局域网,系统启动时通过 NTP 协议同步 UTC 时间,消除设备间的秒级时钟偏差。
    \item \textbf{微秒级打标}:在底层驱动(基于 ESP32-CSI-Tool)中,利用硬件计数器在物理层前导码解析瞬间为每个 CSI 数据包(采样率 120Hz)打上微秒级时间戳。
    \item \textbf{流式汇聚}:CSI 数据通过 UDP 协议实时回传至服务器。服务器端依据全局时间戳对 8 路射频数据流与 CCD 光学数据流进行滑动窗口对齐,有效规避了网络抖动(Jitter)带来的时序误差。
\end{enumerate}

\begin{figure}[htbp]
    \centering
    % \includegraphics[width=0.8\textwidth]{figures/system_layout.png} 
    \caption{分布式8节点感知系统空间布局示意图}
    \label{fig:system_layout}
\end{figure}


% =========================================================================
% 6.3 节:将原“实验设置”与“实验场景设计”整合,形成完整的方法论章节
% =========================================================================
\section{实验环境部署与评估方案}
\label{sec:experiment_setup_evaluation}

基于上述构建的协同感知原型系统,本节将详细阐述实验的具体实施方案。为了全面评估系统在理想环境与复杂室内环境下的定位性能及鲁棒性,我们设计了包含特定被测目标的多维验证场景。

\subsection{被测目标与实验环境配置}
\label{subsec:target_setup}
% [原 6.4.1 内容移动至此]

实验部署于一个典型的室内多径环境(具体布局参数见 \ref{subsec:csi_system})。为了模拟标准点目标的运动特征并确保光电信号的同步可观测性,我们设计了专用的被测目标(Test Target):

\begin{itemize}
    \item \textbf{物理结构}:选用直径 $12\text{cm}$ 的轻质泡沫球作为目标载体。球体表面均匀包裹铝箔材质,以增强对 2.4GHz 射频信号的反射截面积(RCS),确保在 NLOS 场景下仍能产生显著的多径扰动。
    \item \textbf{光学标记}:在球体表面固定特定波长的红外 LED 主动标记点,作为 YOPS 光学系统的追踪靶标。
    \item \textbf{运动控制}:为了最小化人体对射频场的侵入式干扰,实验中采用非金属材质的细长支架(直径 $1\text{cm}$,长 $2\text{m}$)对目标进行悬浮式牵引,确保 CSI 信号的波动主要源于目标球体的运动。
\end{itemize}

\subsection{系统标定与真值坐标系构建}
\label{subsec:system_calibration}
% [原 6.4.2 内容移动至此]

为了确保光电异构数据在同一物理空间下的精确映射,实验前需建立统一的世界坐标系,并完成线阵 CCD 视场的几何标定。本研究采用了**“离线刚体建模—在线激光追踪”**相结合的分级标定策略。

\subsubsection{标定设备与工具}
实验引入了工业级测量设备以构建高精度的 Ground Truth 基准(如图 6.6 所示):
\begin{itemize}
    \item \textbf{高精度关节式三维坐标测量臂(Portable CMM)}:用于建立刚体结构的局部几何特征,提供亚毫米级的接触式测量。
    \item \textbf{激光跟踪仪(Laser 
Tracker)}:作为大范围空间测量的“黄金标准”,用于实时捕捉刚体基座在世界坐标系下的绝对位姿。
\end{itemize}

\begin{figure}[htbp]
    \centering
    % \includegraphics[width=0.8\textwidth]{figures/calibration_devices.png}
    \caption{实验标定设备:(a) 激光跟踪仪; (b) 标记点控制系统;
(c) 关节式三维坐标测量臂}
    \label{fig:calibration_devices}
\end{figure}

\subsubsection{时空基准对齐流程}
标定过程旨在建立 LED 光学中心与世界坐标系之间的刚性变换关系,具体分为两个阶段:

\textbf{1. 离线几何建模(局部坐标系构建)}
在静态环境下,利用三维坐标测量臂对标定板上的 LED 安装节点及靶球基座进行多点接触式采样。通过空间圆拟合算法,解算出各 LED 光心相对于靶球基座中心(Base Center)的局部坐标向量 $\mathbf{P}_{local}$。基于刚体假设,无论标定板如何移动,该相对几何关系始终保持不变。

\textbf{2.在线位姿解算(世界坐标系映射)}
在实验过程中,激光跟踪仪实时锁定靶球基座,获取其在世界坐标系下的位置坐标与姿态四元数。结合离线阶段获取的几何拓扑关系,通过刚体变换矩阵(Rigid Body Transformation Matrix)将 LED 的局部坐标 $\mathbf{P}_{local}$ 映射至全局世界坐标系 $\mathbf{P}_{world}$:
\begin{equation}
    \mathbf{P}_{world} = \mathbf{R} \cdot \mathbf{P}_{local} + \mathbf{T}
\end{equation}
其中 $\mathbf{R}$ 和 $\mathbf{T}$ 分别为由激光跟踪仪测得的旋转矩阵与平移向量。该方法为线阵 CCD 系统提供了具有工业级精度的空间参考点,确保了后续训练数据的标签误差远低于 Wi-Fi 信号的物理分辨率。

\subsection{多维评估场景设计}
\label{subsec:scenarios_design}
% [原 6.5.2 内容移动至此。这是属于“实验方法论”的内容]

针对单一 LOS 场景难以验证系统鲁棒性的问题,本研究设计了以下三组递进式实验场景,旨在从精度上限、遮挡恢复能力及抗干扰能力三个维度对算法进行综合评估。

\subsubsection{场景一:理想视距(LOS)基准测试}
\begin{itemize}
    \item \textbf{实验设置}:移除实验空间内的所有障碍物,确保 CCD 视觉链路与 Wi-Fi 射频链路均无遮挡。
    \item \textbf{采集过程}:实验人员位于长方体空间外部,通过直径 1cm、长 2m 的细长实心木棍控制目标小球,使其以约 $0.1\text{m/s}$ 的速度在空间内部进行缓慢匀速移动。
    \item \textbf{目的}:采集高质量的训练数据构建指纹库,并验证系统在理想条件下的定位精度上限。
\end{itemize}

\subsubsection{场景二:物理约束下的非视距(NLOS)遮挡测试}
\begin{itemize}
    \item \textbf{实验挑战}:在验证 NLOS 性能时,通常面临“视觉真值与射频信号同时被遮挡”的悖论。
    \item \textbf{解决方案}:采用“物理轨迹约束法”。在实验区域中心铺设直线导轨,并在导轨中段设置不透明障碍物(阻断 Wi-Fi 直射路径并遮挡 CCD 视线),使导轨两端(区域 A、B)处于可视区,中间(区域 C)处于盲区。
    \item \textbf{采集过程}:控制目标小球沿导轨匀速穿过遮挡区域。
    \item \textbf{真值构建策略}:虽然 CCD 无法记录遮挡区内的实时坐标,但基于第五章建立的恒定速度(CV)运动模型,利用 CCD 记录的目标进入遮挡区时刻 $t_{start}$ 的坐标 $P_{start}$ 与离开遮挡区时刻 $t_{end}$ 的坐标 $P_{end}$,通过时空线性插值(Spatiotemporal Linear Interpolation)生成遮挡区内的物理真值 $G(t)$:
    \begin{equation} 
       G(t) = P_{start} + \frac{P_{end} - P_{start}}{t_{end} - t_{start}} \times (t - t_{start}), \quad t \in (t_{start}, t_{end})
    \end{equation}
    该方法有效解决了传感器盲区内的真值获取难题,用于评估系统在全遮挡条件下的定位稳定性。
\end{itemize}

\subsubsection{场景三:人机共存的动态干扰测试}
\begin{itemize}
    \item \textbf{实验设置}:保持 CCD 视觉链路通视,引入一名实验人员作为动态干扰源。
    \item \textbf{干扰模式}:干扰者在 Wi-Fi 收发链路附近进行随机走动或切断信号路径,制造强多径效应与环境噪声。
    \item \textbf{目的}:CCD 持续记录受干扰状态下的目标真值,用于验证自适应卡尔曼滤波/粒子滤波算法对突变观测噪声的抑制能力及系统的环境鲁棒性。
\end{itemize}


% =========================================================================
% 6.4 节:专门的“结果分析”章节,框架清晰
% =========================================================================
\section{实验结果分析与性能评估}
\label{sec:results_and_analysis}

\subsection{光学传感器性能验证}
\label{subsec:optical_results}

% [原 6.5.1 内容,此处目前为空,等待填充数据]
% 建议内容:YOPS 的静态定位精度(均方根误差 RMSE)、动态延迟测试结果等。


\subsection{协同定位系统综合性能评估}
\label{subsec:system_results}

% [此处对应原 6.5.2 的结果部分,目前建立框架]

\subsubsection{理想场景下的精度上限分析}
% 对应场景一的结果。展示 LOS 下的 CDF 图,对比纯 CSI 和 光电协同 的精度。

\subsubsection{遮挡恢复与 NLOS 鲁棒性分析}
% 对应场景二的结果。展示在导轨遮挡区,纯 CSI 可能会漂移,而协同系统利用 RF 备份模式维持了多少精度。

\subsubsection{动态环境干扰下的稳定性分析}
% 对应场景三的结果。展示有人走动时,定位误差的波动情况。

\section{本章小结}
% 总结本章工作:实现了硬件,设计了严谨的真值获取方案,并通过实验验证了系统的优越性。