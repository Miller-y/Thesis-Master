% !Mode:: "TeX:UTF-8"
\chapter{线阵视觉系统的几何畸变校正与高鲁棒性标定方法实现}
\label{chap:calibration}

\section{引言}

基于视觉的三维测量技术通常建立在理想的几何光学模型之上。然而,在实际的物理成像系统中,由于制造工艺与装配精度的限制,镜头焦距的一致性、相机在空间中的绝对位姿、以及传感器与光学中心的耦合关系等,都不可避免地存在偏差。这些物理参数的非理想特性是引入测量误差的主要来源。因此,为了实现高精度的空间目标检测,必须对系统进行严格的标定与校准。

\subsection{现有技术现状与局限}

在相机标定领域,针对面阵相机的研究已相当成熟,学术界和工业界发展出了多种经典算法。相比之下,针对线阵相机的标定技术研究相对匮乏。尽管文献中提出了适用于线阵模型的“7系数DLT方法”,为该领域提供了一定的理论基础,但在面对复杂光学系统时,其适用性仍面临挑战。

在本系统所采用的硬件架构中,为了满足特定的视场覆盖需求,引入了组合柱面透镜系统。实际测试表明,该光学结构引入了显著的非线性几何畸变,特别是在偏离光轴较大的边缘视场区域。若直接套用传统的线性模型,忽略镜头畸变带来的非线性误差,将导致标定结果偏离真实值。因此,如何有效地分离并校正这种特殊的镜头畸变,并克服远距离观测下的噪声干扰,成为本系统标定的关键难点。

\subsection{分阶段标定策略}

针对上述复杂光学系统带来的非线性畸变与远场观测噪声分布不均等挑战,若直接采用整体优化方法求解,极易陷入局部最优。为此,本文提出一种“先校正、后解算”的分阶段解耦标定策略,将复杂的标定问题分解为两个相对独立的子问题进行处理:

\paragraph{第一阶段:非线性几何畸变校正}
该阶段旨在剥离柱面透镜组引入的非线性误差。本文利用 ZEMAX 光学仿真软件生成高精度的栅格畸变数据,从而获取空间中点光源的“真实成像位置”与“理想成像位置”之间的偏差信息。 基于这些仿真数据,在 MATLAB 中建立高阶多项式畸变修正模型,并计算相应的畸变修正系数。此步骤的核心在于构建从“畸变图像坐标系”到“理想图像坐标系”的映射关系,从而消除光学系统非线性特性的影响。

\paragraph{第二阶段:线性参数鲁棒解算}
在完成畸变校正后,后续的参数估计可被视为线性模型下的估计问题。首先利用第一阶段获取的校正系数对所有标志点的原始信号进行预处理,将其恢复为符合针孔/线阵成像模型的理想坐标。随后,针对远距离观测中存在的噪声异方差问题,引入加权估计策略(详见后文 2.3 节),精确解算出系统的线性结构系数及相机的内外部参数。通过这种分步策略,有效规避了非线性畸变与几何投影参数之间的强耦合,显著提升了标定结果的数值稳定性与精度。

\section{柱面镜头的畸变与去畸变}
\label{sensor_distortion}
\subsection{畸变模型}
本文所用柱面镜头组由光学设计软件 ZEMAX 设计,通过 ZEMAX 仿真,可以得到柱面镜头组理论上的畸变结果。ZEMAX 生成的栅格畸变数据包括空间中不同位置的点光源对应的真实成像位置 $\tilde{x}$ 和理想成像位置 $x$,单位为 mm,位于图像坐标系。如图\ref{fig:distortion_with_𝛼_𝛽}所示,畸变为真实位置和理想位置的差,$\alpha$ 表示点光源和相机坐标系 $Y$ 轴所在平面与相机坐标系 $Z$ 轴的夹角,$\beta$ 表示点光源和相机坐标系 $X$ 轴所在平面与相机坐标系 $Z$ 轴的夹角,$\alpha$ 和 $\beta$ 的范围为 $[-20.8^\circ, 20.8^\circ]$。

\begin{figure}[htbp]
    \centering
    \includegraphics[width=0.6\textwidth]{figures/chap05/畸变和alpha_beta之间的关系.pdf}
    \caption{畸变和𝛼、𝛽之间的关系}
    \label{fig:distortion_with_𝛼_𝛽} % 对应文中的 \ref{fig:distortion_with_𝛼_𝛽}
\end{figure}

假设空间中的点光源在相机坐标系下的坐标为 $^C \boldsymbol{M} = [X_C \quad Y_C \quad Z_C]^T$,定义归一化平面 $[x_n \quad y_n \quad 1]^T$,其中:
\begin{equation}
x_n = X_C / Z_C = \tan \alpha \tag{3.1}
\end{equation}
\begin{equation}
y_n = Y_C / Z_C = \tan \beta \tag{3.2}
\end{equation}

将镜头的相对畸变 $\Delta$ 表示为 $x_n$ 和 $y_n$ 的函数,定义如下:
\begin{equation}
\Delta = 
\begin{cases} 
\frac{\tilde{x} - x}{x} = \frac{\tilde{x}_n - x_n}{x_n} = h(x_n, y_n), & x_n \neq 0 \\
0, & x_n = 0
\end{cases} \tag{3.3}
\end{equation}

其中带有上标变量 $(\tilde{\cdot})$ 表示存在畸变的真实值,无上标变量表示理想值,$h(\cdot)$ 表示关于 $x_n$ 和 $y_n$ 的函数。根据 $\Delta$ 和 $x_n$、$y_n$ 之间的关系可以得到新的曲面,如图\ref{fig:distortion_with_𝑥𝑛_𝑦𝑛}所示。


\begin{figure}[htbp]
    \centering
    \includegraphics[width=0.6\textwidth]{figures/chap05/相对畸变和xn_yn之间的关系.pdf}
    \caption{相对畸变和𝑥𝑛、𝑦𝑛之间的关系}
    \label{fig:distortion_with_𝑥𝑛_𝑦𝑛} % 对应文中的 \ref{fig:distortion_with_𝑥𝑛_𝑦𝑛}
\end{figure}


本文利用 MATLAB 中的曲线拟合工具箱,使用多项式对图中曲面进行拟合,结果显示将 $h(x_n, y_n)$ 表示为四阶多项式的形式可以较好地拟合曲面,同时奇数次幂的系数可以忽略不计。因此可以将 $h(x_n, y_n)$ 定义如下:
\begin{equation}
h(x_n, y_n) = p_1 x_n^2 + p_2 y_n^2 + p_3 x_n^2 y_n^2 + p_4 x_n^4 + p_5 y_n^4 \tag{3.4}
\end{equation}

根据公式 (3.3) 可得 $\tilde{x}_n$ 和 $x_n$ 之间的关系为:
\begin{equation}
\tilde{x}_n = x_n (1 + h(x_n, y_n)) \tag{3.5}
\end{equation}

其中 $\tilde{x}_n$ 的值可由公式 (3.3) 计算得到。使用 MATLAB 中 \texttt{lsqcurvefit} 函数进一步拟合公式 (3.5),各个系数的拟合结果如表 3.1 所示,其中均方根误差(Root Mean Square Error, RMSE)为 $0.0052 \mu\text{m}$。

\begin{table}[h]
\centering
\caption{表 3.1 $h(x_n, y_n)$ 各项系数拟合结果}
\begin{tabular}{cccccc}
\hline
$p_1$ & $p_2$ & $p_3$ & $p_4$ & $p_5$ & RMSE/$\mu\text{m}$ \\ \hline
-0.1134 & -0.0258 & 0.2047 & 0.1727 & 0.0189 & 0.0052 \\ \hline
\end{tabular}
\end{table}

结合公式 (3.1) 和 (3.5),可得该柱面镜头组的畸变模型为:
\begin{equation}
\tilde{u} = f (1 + h(x_n, y_n)) x_n + u_0, \quad x_n = \frac{u - u_0}{f} \tag{3.6}
\end{equation}

然而,由公式 (2.32) 可知,$y_n$ 无法被线阵相机观测,因此需要将模型简化。

\begin{figure}[htbp]
    \centering
    \includegraphics[width=0.6\textwidth]{figures/chap05/不同alpha的情况下beta变化导致的畸变变化.pdf}
    \caption{不同𝛼的情况下𝛽变化导致的畸变变化}
    \label{fig:𝛽_with_𝛼_change} % 对应文中的 \ref{fig:𝛽_with_𝛼_change}
\end{figure}

图\ref{fig:𝛽_with_𝛼_change}展示了在不同 $\alpha$ 的情况下,仅 $\beta$ 变化导致的畸变变化。图\ref{fig:max_𝛽_with_𝛼_change}中横坐标表示不同的 $\alpha$,纵坐标表示在当前 $\alpha$ 保持不变的情况下,$\beta$ 变化导致的像点最大偏移。由图 3.4 可知,当 $\alpha$ 在 $\pm 10.8^\circ$ 左右时,$\beta$ 变化导致的最大偏移达到最大值。

\begin{figure}[htbp]
    \centering
    \includegraphics[width=0.6\textwidth]{figures/chap05/beta变化导致的像点最大偏移.pdf}
    \caption{𝛽变化导致的像点最大偏移}
    \label{fig:max_𝛽_with_𝛼_change} % 对应文中的 \ref{fig:max_𝛽_with_𝛼_change}
\end{figure}


当 $\|\beta\| \leq 20.8^\circ$ 时,最大值为 $17.9 \mu\text{m}$,当 $\|\beta\| \leq 15^\circ$ 时,最大值为 $8.42 \mu\text{m}$。又因为本文所用线阵 CCD 的像素单元长度为 $8 \mu\text{m}$,$\beta$ 变化导致的像点偏移在大多数情况下不超过 1 个像素点,所以可假设 $\beta$ 变化产生的畸变对于测量精度影响不大。

将 $h(x_n, y_n)$ 重新定义为只和 $x_n$ 相关的多项式函数,则 $\tilde{x}_n$ 和 $x_n$ 之间的关系可表示为:
\begin{equation}
\tilde{x}_n = x_n (1 + h(x_n)) \tag{3.7}
\end{equation}

利用 MATLAB 中 \texttt{lsqcurvefit} 函数,使用四阶多项式对公式 (3.7) 拟合,$h(x_n)$ 的拟合结果如下:
\begin{equation}
h(x_n) = \frac{\tilde{x}_n - x_n}{x_n} = p_1 x_n^2 + p_2 x_n^4 \tag{3.8}
\end{equation}

其中各个系数如表 3.2 所示,拟合结果的 RMSE 为 $0.1287 \mu\text{m}$。

\begin{table}[h]
\centering
\caption{表 3.2 $h(x_n)$ 各项系数拟合结果}
\begin{tabular}{ccc}
\hline
$p_1$ & $p_2$ & RMSE/$\mu\text{m}$ \\ \hline
-0.1325 & 0.3228 & 0.1287 \\ \hline
\end{tabular}
\end{table}

简化后的柱面镜头组的畸变模型可表示为:
\begin{equation}
\tilde{u} = f x_n (1 + h(x_n)) + u_0, \quad x_n = \frac{u - u_0}{f} \tag{3.9}
\end{equation}

\subsection{去畸变模型}

去畸变的过程是由实际像素坐标 $\tilde{u}$ 计算理想像素坐标 $u$ 的过程。为了得到去畸变模型,本文将相对畸变重新定义为:
\begin{equation}
\Delta' = 
\begin{cases} 
\frac{x - \tilde{x}}{\tilde{x}} = \frac{x_n - \tilde{x}_n}{\tilde{x}_n} = g(\tilde{x}_n), & \tilde{x}_n \neq 0 \\
0, & \tilde{x}_n = 0
\end{cases} \tag{3.10}
\end{equation}

利用 MATLAB 中的曲线拟合工具箱,使用多项式对公式 (3.10) 进行拟合,结果显示 $g(\tilde{x}_n)$ 可同样表示为四阶多项式的形式,同时奇数次幂的系数可以忽略不计。$g(\tilde{x}_n)$ 拟合结果如下:
\begin{equation}
g(\tilde{x}_n) = k_1 \tilde{x}_n^2 + k_2 \tilde{x}_n^4 \tag{3.11}
\end{equation}

其中各个系数如表 3.3 所示,拟合结果的 RMSE 为 $0.1289 \mu\text{m}$。

\begin{table}[h]
\centering
\caption{表 3.3 $g(\tilde{x}_n)$ 各项系数拟合结果}
\begin{tabular}{ccc}
\hline
$k_1$ & $k_2$ & RMSE/$\mu\text{m}$ \\ \hline
0.1349 & -0.3252 & 0.1289 \\ \hline
\end{tabular}
\end{table}

图像去畸变模型可表示为:
\begin{equation}
u = G(\tilde{u}) = f \tilde{x}_n (1 + g(\tilde{x}_n)) + u_0, \quad \tilde{x}_n = \frac{\tilde{u} - u_0}{f} \tag{3.12}
\end{equation}




\section{考虑几何灵敏度的线阵相机参数加权估计方法}

\subsection{问题背景与方法提出}

经典的直接线性变换(DLT)标定方法在近场、高信噪比条件下通常能保持较好的稳定性。然而,在本文所涉及的 1m 至 6m 大工作距离范围以及复杂光照环境中,观测噪声往往呈现出显著的异方差特性(Heteroscedasticity)。若直接采用传统 DLT 方法并通过“差分消去”构建方程,极易导致远距离特征点的噪声耦合与误差放大,从而降低参数估计的最终精度。

在本文所涉及的 1m 至 6m 大工作距离范围以及复杂光照环境中,直接线性变换(DLT)虽然通过代数线性化解决了模型非凸、初值难寻的计算难题,但也必然引入了新的统计偏差。这种偏差源于线性化过程中引入的射影深度因子,它人为地将观测噪声与空间距离耦合,导致系统表现出显著的异方差特性(Heteroscedasticity)。

具体而言,在长景深观测条件下,远距离点(大深度)的代数残差会被几何因子显著放大。若不加干预,优化算法将过度拟合远处的低信噪比数据,而牺牲近处高精度区域的准确性。这并非单纯的物理噪声问题,而是数学模型线性化带来的结构性缺陷。 针对这一“线性化代价”,本节引入统计信号处理中的加权估计思想,提出一种基于几何灵敏度与测量不确定性的加权最大似然估计(Weighted Maximum Likelihood Estimation, WMLE)方法。该方法主要包含三个核心策略:首先,摒弃传统的差分构方,转而采用交叉相乘残差构建目标函数,以避免被减项选择带来的数值不稳定性;其次,在最大似然估计的框架下,通过误差传播分析推导残差方差,从而定义统计意义上的最优权重;最后,结合鲁棒代价函数与工程先验,构建完整的非线性最小二乘标定流程。

\subsection{加权目标函数的严格推导}

\subsubsection{成像模型与交叉相乘残差}

基于前文所述的七参数有理投影模型,线阵相机的一维投影关系通常描述为像点坐标 $u_j$ 是空间点坐标 $(X_j, Y_j, Z_j)$ 的分式函数:
\begin{equation}
u_j = \frac{l_1 X_j + l_2 Y_j + l_3 Z_j + l_4}{l_5 X_j + l_6 Y_j + l_7 Z_j + 1}
\label{eq:rational_model}
\end{equation}
为了简化书写并明确物理意义,我们将分子项 $N_j$ 与分母项 $D_j$ 分别定义为:
\begin{align}
N_j &:= l_1 X_j + l_2 Y_j + l_3 Z_j + l_4 \\
D_j &:= l_5 X_j + l_6 Y_j + l_7 Z_j + 1
\end{align}
其中,分母项 $D_j$ 在几何上对应于空间点在相机光轴方向上的投影深度(即射影深度因子)。

直接使用公式 (\ref{eq:rational_model}) 进行参数估计面临非线性强、奇异点敏感等问题。鉴于此,我们将分式形式转化为交叉相乘(Cross-Product)形式,即等式两边同乘分母 $D_j$,得到关于待求参数 $\boldsymbol{p} = [l_1, \dots, l_7]^\top$ 的线性约束条件:
\begin{equation}
N_j - u_j D_j = 0
\label{eq:cross_product_expanded}
\end{equation}
展开后可得:
\begin{equation}
(l_1 X_j + l_2 Y_j + l_3 Z_j + l_4) - u_j (l_5 X_j + l_6 Y_j + l_7 Z_j + 1) = 0
\end{equation}
这一变换具有双重优势:其一,它消除了直接除法可能引入的数值不稳定问题(尤其是当点位于无穷远平面附近时);其二,它将原始的非线性模型转化为对参数线性的代数误差形式,为后续的高效初始化提供了可能。

\subsubsection{误差传播与权重设计}

设像素坐标的真实值为 $u_j^*$,观测值为 $u_j$,假设观测噪声服从零均值高斯分布:
\begin{equation}
u_j = u_j^* + \delta u_j, \quad \delta u_j \sim \mathcal{N}(0, \sigma_{u,j}^2)
\end{equation}
定义交叉相乘残差函数 $r_j(\boldsymbol{p}, u_j) = N_j - u_j D_j$。将观测模型代入该式,并利用一阶泰勒展开进行近似:
\begin{equation}
r_j = N_j - (u_j^* + \delta u_j) D_j = (N_j - u_j^* D_j) - \delta u_j D_j \approx - \delta u_j D_j
\end{equation}
由此可推导出残差的方差为:
\begin{equation}
\text{Var}[r_j] \approx \text{Var}[-\delta u_j D_j] = \sigma_{u,j}^2 D_j^2
\end{equation}
上述推导表明,代数误差的方差$\text{Var}[r_j]$与投影深度$D_j$ 的平方成正比。这一结论揭示了 DLT 方法在远场标定中失效的数学本质:为了获得闭式解而引入的交叉相乘变换,在数学上等效于对不同距离的观测点施加了非均匀的“噪声增益”。

因此,这种异方差性是线性化求解策略的固有副产物。 为了在保留线性求解高效性的同时消除其副作用,必须构建与 $D_j$ 逆相关的统计权重。最大似然估计(MLE)框架下的加权最小二乘问题可表述为:
\begin{equation}
\min_{\boldsymbol{p}} J(\boldsymbol{p}) = \sum_{j=1}^{N} w_j r_j(\boldsymbol{p})^2
\end{equation}
其中,统计最优权重 $w_j$ 定义为:
\begin{equation}
w_j = \frac{1}{\sigma_{u,j}^2 D_j^2}
\end{equation}
该权重设计自然编码了成像几何灵敏度与测量不确定性:距离相机越远的点($D_j$ 越大),其权重越低,这与物理直觉完全一致。

\subsection{非线性优化与实施流程}

\subsubsection{优化策略}

针对上述模型,本文采用“两步法”优化策略,即首先通过线性代数方法获取高质量的参数初值,随后利用非线性迭代算法精化结果,以兼顾计算效率与估计精度。

\paragraph{1. 线性初始化 (DLT/SVD)}
基于交叉相乘公式 (\ref{eq:cross_product_expanded}),对于 $N$ 个观测点,我们可以构建如下形式的齐次线性方程组:
\begin{equation}
\boldsymbol{A} \boldsymbol{p} = \boldsymbol{0}
\end{equation}
其中数据矩阵 $\boldsymbol{A} \in \mathbb{R}^{N \times 7}$ 的第 $j$ 行向量 $\boldsymbol{a}_j$ 为:
\begin{equation}
\boldsymbol{a}_j = \left[ X_j, Y_j, Z_j, 1, -u_j X_j, -u_j Y_j, -u_j Z_j \right]
\end{equation}
为了避免零解(Trivial Solution),通常引入约束 $\|\boldsymbol{p}\| = 1$。此时,该问题的求解转化为寻找矩阵 $\boldsymbol{A}$ 的最小奇异值对应的右奇异向量。我们利用奇异值分解(SVD)对 $\boldsymbol{A}$ 进行分解,取其最后一行右奇异向量作为参数向量的初始估计值 $\boldsymbol{p}_0$。此步骤即为经典的直接线性变换(DLT)解法。

\paragraph{2. 加权非线性迭代 (LM算法)}
线性解虽然计算简便,但其最小化的是代数误差而非几何误差,且未考虑各观测点的噪声异方差性。因此,我们将 $\boldsymbol{p}_0$ 作为初值,采用 Levenberg-Marquardt (LM) 算法对加权平方残差和目标函数 $J(\boldsymbol{p})$ 进行非线性最小化:
\begin{equation}
\min_{\boldsymbol{p}} J(\boldsymbol{p}) = \sum_{j=1}^{N} w_j \left( N_j(\boldsymbol{p}) - u_j D_j(\boldsymbol{p}) \right)^2
\end{equation}
在第 $k$ 次迭代中,我们需要求解如下形式的增量正规方程(Augmented Normal Equations)以获取参数更新量 $\Delta \boldsymbol{p}$:
\begin{equation}
(\boldsymbol{J}^\top \boldsymbol{W} \boldsymbol{J} + \lambda \boldsymbol{I}) \Delta \boldsymbol{p} = -\boldsymbol{J}^\top \boldsymbol{W} \boldsymbol{r}
\end{equation}
其中:
\begin{itemize}
    \item $\boldsymbol{J}$ 为残差向量关于参数的雅可比矩阵;
    \item $\boldsymbol{W} = \text{diag}(w_1, \dots, w_N)$ 为基于测量不确定性构建的权重矩阵;
    \item $\lambda$ 为阻尼因子,用于在梯度下降法(大 $\lambda$)与高斯-牛顿法(小 $\lambda$)之间动态调节,确保算法在远离极值点时的稳定性和接近极值点时的收敛速度。
\end{itemize}
通过不断更新 $\boldsymbol{p} \leftarrow \boldsymbol{p} + \Delta \boldsymbol{p}$ 直至收敛,即可得到几何意义下最优的相机参数。

\subsubsection{测量不确定性的估计与算法实现}

为了准确计算权重,测量不确定性 $\sigma_{u,j}$ 的估计至关重要。本文采用重复成像法,即对每个空间标记点采集多帧图像(如 10 帧),计算其亚像素坐标的标准差作为 $\sigma_{u,j}$ 的估计值。

综上所述,WMLE 线阵相机标定算法的完整实施流程如下:

\begin{enumerate}
    \item \textbf{数据归一化}:对输入数据 $(X_j, Y_j, Z_j)$ 和 $u_j$ 进行中心化与尺度归一化处理,以提高数值稳定性,并记录逆变换矩阵。
    \item \textbf{线性初始化}:基于归一化数据构建线性方程组,利用 SVD 分解获取参数初值 $\boldsymbol{p}_0$。
    \item \textbf{迭代优化 (LM算法)}:
    \begin{itemize}
        \item 更新几何因子:$D_j = l_5 X_j + l_6 Y_j + l_7 Z_j + 1$;
        \item 更新权重:根据 $w_j = 1/(\sigma_{u,j}^2 D_j^2)$ 计算各点权重;
        \item 计算残差与雅可比:构建加权残差向量与加权雅可比矩阵;
        \item 求解增量并更新:求解阻尼最小二乘问题,更新参数 $\boldsymbol{p}$ 及阻尼系数 $\lambda$;
        \item 收敛判断:若参数增量或残差下降率小于阈值,则停止迭代。
    \end{itemize}
    \item \textbf{参数恢复}:利用步骤 1 中的逆变换矩阵,将优化后的参数恢复至原始物理坐标系,输出最终标定结果及其协方差估计。
\end{enumerate}

该方法通过合理引入误差传播分析,在保证计算复杂度的前提下,有效提升了线阵视觉系统在复杂工程条件下的标定鲁棒性。




\section{相机内参计算方法}

定义线阵相机内参矩阵为:
\begin{equation}
\mathbf{K} = \begin{bmatrix} f & u_0 \\ 0 & 1 \end{bmatrix}
\label{eq:intrinsic_def}
\end{equation}

定义矩阵 $\mathbf{B}$ 为:
\begin{equation}
\begin{aligned}
\mathbf{B} &= \begin{bmatrix} \mathbf{b}_1 \\ \mathbf{b}_2 \end{bmatrix} = \begin{bmatrix} f & u_0 \\ 0 & 1 \end{bmatrix}^{-1} \mathbf{L} = \begin{bmatrix} \frac{1}{f} & -\frac{u_0}{f} \\ 0 & 1 \end{bmatrix} \mathbf{L} \\
&= \begin{bmatrix} \frac{l_1 - u_0 l_5}{f} & \frac{l_2 - u_0 l_6}{f} & \frac{l_3 - u_0 l_7}{f} & \frac{l_4 - u_0}{f} \\ l_5 & l_6 & 1 \end{bmatrix}
\end{aligned}
\label{eq:matrix_B}
\end{equation}
其中 $\lambda$ 为任意缩放因子。

由于 $\mathbf{R}$ 是正交矩阵,其行向量的模均为 1,且任意两行不同向量的内积为 0。由 $\mathbf{r}_1 \cdot \mathbf{r}_3 = 0$ 可得如下关系:
\begin{equation}
\mathbf{b}_1 \mathbf{b}_2^T = \lambda^2 \mathbf{r}_1 \mathbf{r}_3^T = 0
\label{eq:orthogonality_constraint}
\end{equation}

进一步整理可得:
\begin{equation}
\left(l_5^2 + l_6^2 + 1\right) u_0 = l_1 l_5 + l_2 l_6 + l_3
\label{eq:solve_u0}
\end{equation}

由 $\mathbf{r}_1 \mathbf{r}_1^T = \mathbf{r}_3 \mathbf{r}_3^T = 1$ 可得如下关系:
\begin{equation}
\mathbf{b}_1 \mathbf{b}_1^T = \mathbf{b}_2 \mathbf{b}_2^T
\label{eq:norm_constraint}
\end{equation}

结合公式(\ref{eq:solve_u0}),进一步整理可得:
\begin{equation}
l_1^2 + l_2^2 + l_3^2 - \left(l_1 l_5 + l_2 l_6 + l_3\right) u_0 = \left(l_5^2 + l_6^2 + 1\right) f^2
\label{eq:solve_f}
\end{equation}

获取空间中少量(数量大于 7 个)固定的标记点对应的像点,用上一小节的方法可计算出投影矩阵 $\mathbf{L}$。根据公式(\ref{eq:solve_u0})可计算出 $u_0$,之后根据已知 $u_0$ 和公式(\ref{eq:solve_f})可计算出 $f$。

需要有效确保像点在线阵 CCD 上均匀分布。

\section{相机坐标系和世界坐标系之间的刚体变换}

每个投影矩阵 $\mathbf{L}$ 都对应了一组 $[\mathbf{R} \mid \mathbf{t}]$。相机坐标系 $\mathcal{F}_c$ 和世界坐标系 $\mathcal{F}_w$ 之间的旋转矩阵 $\mathbf{R}$ 和平移向量 $\mathbf{t}$ 可由上一节算出的相机内参和对应的 $\mathbf{L}$ 计算。结合前述公式可得旋转向量:
\begin{equation}
\mathbf{r}_1 = \frac{\left[ l_1 - u_0 l_5,\ \ l_2 - u_0 l_6,\ \ l_3 - u_0 l_7 \right]}{\left\| \left[ l_1 - u_0 l_5,\ \ l_2 - u_0 l_6,\ \ l_3 - u_0 l_7 \right] \right\|}
\label{eq:solve_r1}
\end{equation}

\begin{equation}
\mathbf{r}_3 = \frac{\left[ l_5,\ \ l_6,\ \ l_7 \right]}{\left\| \left[ l_5,\ \ l_6,\ \ l_7 \right] \right\|}
\label{eq:solve_r3}
\end{equation}

\begin{equation}
\mathbf{r}_2 = \mathbf{r}_3 \times \mathbf{r}_1
\label{eq:solve_r2}
\end{equation}

平移向量的分量计算如下:
\begin{equation}
t_z = \frac{1}{\left\| \left[ l_5,\ \ l_6,\ \ l_7 \right] \right\|}
\label{eq:solve_tz}
\end{equation}

\begin{equation}
t_x = \frac{l_4 - u_0}{f} \cdot t_z
\label{eq:solve_tx}
\end{equation}

平移向量的 $Y$ 轴部分 $t_y$ 无法计算,将其设定为 0,这和配有柱面镜的线阵相机特性相符,即在相机坐标系下,若空间中一标记点的 $X$、$Z$ 轴方向坐标不变,$Y$ 轴方向上的变化不会导致像点位置变化。

由于噪声和镜头畸变的影响,计算出的 $\mathbf{R}$ 不一定满足标准旋转矩阵的所有约束(如正交性),通常需要使用 SVD 重新计算新的旋转矩阵以确保其处于 $SO(3)$ 空间内。




% =============================================================================
% 需要在导言区(preamble)添加的宏包:
% \usepackage{amsmath}
% \usepackage{graphicx}
% \usepackage{booktabs} % 用于三线表 \toprule, \midrule, \bottomrule
% \usepackage{multirow} % 用于表格合并行
% \usepackage{float}    % 用于控制浮动体位置 [H]
% =============================================================================

\section{仿真实验与分析}
\label{sec:simulation_experiment}

为了验证本文提出的Y型三线阵CCD传感器设计参数的合理性,以及加权最大似然估计(WMLE)标定算法在复杂噪声环境下的鲁棒性,本章基于Matlab平台构建了完整的仿真实验环境。实验内容主要包含传感器视域(FOV)分析、标定算法抗噪性能验证以及多相机系统定位精度评估三个部分。

\subsection{Y型CCD传感器视域仿真分析}

\subsubsection{仿真模型构建}
为了验证本文设计的Y型三线阵CCD传感器的有效探测范围,并指导后续实际系统的安装调试,本文构建了传感器视场(Field of View, FOV)仿真模型。

考虑到室内导航环境的层高(通常在 $3\,\text{m}$ 至 $5\,\text{m}$ 之间)以及对水平测量范围的需求,具体的仿真参数设置如表 \ref{tab:sensor_sim_params} 所示。

\begin{table}[htbp]
    \centering
    \caption{传感器视域仿真参数配置}
    \label{tab:sensor_sim_params}
    \renewcommand{\arraystretch}{1.2} % 增加行高
    \begin{tabular}{cccc}
        \toprule
        \textbf{参数名称} & \textbf{符号} & \textbf{数值} & \textbf{说明} \\
        \midrule
        传感器型号 & - & Toshiba TCD1304 & 高灵敏度线阵 CCD \\
        像元间距 & $d_{pixel}$ & $8\,\mu \text{m}$ & 硬件固有参数 \\
        有效像素数 & $N$ & 3648 & 硬件固有参数 \\
        CCD有效长度 & $L_{total}$ & $29.184\,\text{mm}$ & $N \times d_{pixel}$ \\
        镜头焦距 & $f$ & \textbf{35.00 mm} & 广角设计,提升覆盖范围 \\
        透镜中心高度 & $H_{lens}$ & $50.00\,\text{mm}$ & 相对CCD平面的垂直高度 \\
        安装半径 & $R$ & \textbf{80.00 mm} & 透镜光心至系统中心的水平距离 \\
        \bottomrule
    \end{tabular}
\end{table}

相较于部分文献采用的 $f=50\,\text{mm}$ 方案,本文选用 $f=35\,\text{mm}$ 镜头显著增大了单视场张角;同时将安装半径设定为 $80\,\text{mm}$,在保证结构紧凑性的同时,优化了三个视场的重叠区域起始高度。

仿真计算区间设定为垂直高度 $Z \in [300, 6000]\,\text{mm}$。在这一高度范围内,计算三个CCD传感器的独立视场,并求解它们的空间交集(Intersection)。该交集即为系统能够同时捕捉到发射源信号的有效导航区域。仿真结果如图 \ref{fig:fov_simulation} 所示。

\begin{figure}[!htbp]
  \centering
  \bisubcaptionbox{整体视图}{Overview}[0.45\textwidth]{
    \includegraphics[width=\linewidth]{figures/chap05/FOV.pdf}
  }
%   \hfill
  \bisubcaptionbox{俯视图}{Planform}[0.45\textwidth]{
    \includegraphics[width=\linewidth]{figures/chap05/FOV-up.pdf}
  }
  \bicaption{Y型三线阵CCD传感器有效视域(FOV)仿真}{Y-type three-line CCD sensor effective field of view (FOV) simulation}
  \label{fig:fov_simulation}
\end{figure}

从仿真结果可以得出以下结论:
\begin{enumerate}
    \item \textbf{视场形态特征}:系统的有效视场在空间中呈现倒置的三棱锥状(或花瓣状)结构。随着高度 $Z$ 的增加,三个独立视场的重叠面积逐渐扩大,有效探测范围显著增加。
    \item \textbf{盲区分析}:在高度 $Z < 300\,\text{mm}$ 的近场区域,由于三个传感器的安装间距($R=80\,\text{mm}$)以及视场角的限制,存在一个倒圆锥形的探测盲区。这表明发射源(LED)与接收机必须保持一定的最小垂直距离,系统才能正常解算坐标。
    \item \textbf{覆盖范围验证}:在典型室内天花板高度(例如 $Z=3000\,\text{mm}$ 处),系统的有效视场覆盖半径约为 $\pm 1436\,\text{mm}$ 的范围。这证明了本文选用的 $35\,\text{mm}$ 焦距参数能够满足室内移动目标的广域定位需求,验证了光学参数设计的合理性。
\end{enumerate}

该仿真结果为后续实验平台的搭建提供了理论依据,确保了在预定的一般室内层高下,系统具有足够的有效捕捉范围。

\subsection{基于去畸变残差模型的标定算法鲁棒性验证}

为了验证本文提出的加权最大似然估计(WMLE)算法在真实光学系统中的适用性,本节构建了与上一小节设计严格一致的仿真实验。

\subsubsection{异方差噪声模型构建}
本实验的核心在于构建符合物理规律的噪声模型。根据第 \ref{sensor_distortion} 节(注:请确保您的论文中有对应的label)分析,柱面透镜组在非近轴区域存在显著的光学畸变。仿真数据显示,光线入射角 $\alpha$(对应CCD像素坐标 $u$ 的变化)对畸变贡献最大,在视场边缘 $\pm 20.8^\circ$ 处,几何畸变可达 $0.2\,\text{mm}$(约25个像素)。尽管系统设计了去畸变模型进行预处理,但受限于模型阶数和加工装配误差,无法完全消除所有高阶非线性畸变。

因此,实验构建了如下的\textbf{异方差残差模型}来模拟“去畸变后的残留误差”:
\begin{equation}
    \sigma_{total} = \sigma_{readout} + k_{\alpha} \cdot \left| \frac{u - u_{c}}{u_{max}} \right|^3 + k_{\beta} \cdot \left( \frac{Z}{Z_{max}} \right)^2
\end{equation}
其中,$\sigma_{readout}$ 为传感器固有的读出噪声(设为0.1像素);第二项模拟视场边缘未校正彻底的高阶畸变残差,呈立方增长趋势;第三项模拟远距离处的像散与光斑离焦效应。实验将基准噪声水平(Noise Level)从 0 扩展至 4.0 像素,以覆盖从理想工况到去畸变失效(高残差)的极端场景。

\subsubsection{标定精度的对比分析}
为了全面评估算法的性能,本实验引入了两个互补的评价指标:\textbf{测量残差(Measurement Residual)}与\textbf{真实几何误差(True Geometric Error)}。

\paragraph{1. 测量残差 (Fitting Residual)}
测量残差反映了标定模型对\textbf{观测数据}的拟合程度,其均方根误差(RMSE)定义为:
\begin{equation}
    E_{residual} = \sqrt{\frac{1}{N} \sum_{i=1}^{N} \|\pi(\boldsymbol{X}_i, \hat{\boldsymbol{L}}) - \boldsymbol{u}_{meas}^{(i)}\|^2}
\end{equation}
其中,$\boldsymbol{u}_{meas}^{(i)}$ 为包含异方差噪声的第 $i$ 个实际观测像点,$\hat{\boldsymbol{L}}$ 为算法估计出的投影矩阵。

\paragraph{2. 真实几何误差 (True Geometric Error)}
真实几何误差反映了标定模型对\textbf{系统真值}的还原能力。由于仿真环境下系统的真实投影矩阵 $\boldsymbol{L}_{true}$ 与无噪声的理想像点 $\boldsymbol{u}_{true}^{(i)}$ 已知,该指标定义为:
\begin{equation}
    E_{true} = \sqrt{\frac{1}{N} \sum_{i=1}^{N} \|\pi(\boldsymbol{X}_i, \hat{\boldsymbol{L}}) - \boldsymbol{u}_{true}^{(i)}\|^2}
\end{equation}

实验对比了归一化DLT、普通LM优化与WMLE算法在恢复投影矩阵 $\boldsymbol{L}$ 及内参焦距 $f$ 时的精度。每组噪声水平下进行100次蒙特卡洛独立实验。

\textbf{值得注意的是},在均匀高斯噪声下,$E_{residual}$ 的降低通常意味着 $E_{true}$ 的同步改善。但在本实验设定的异方差噪声环境下,若算法过度拟合了高噪声的边缘数据,可能导致模型偏离物理真值。这种“残差主要由离群点主导,精度主要由模型内参决定”的矛盾,是评判算法鲁棒性的关键。

\paragraph{(1) 对物理残差的抑制能力(重投影精度)}
图 \ref{fig:reproj_comparison} (a) 和 (b) 分别展示了测量残差和真实几何误差的变化趋势,具体数据见表 \ref{tab:reproj_error_comparison}。

\begin{figure}[!htbp]
  \centering
  \bisubcaptionbox{测量残差对比}{Residual comparison}[0.45\textwidth]{
    \includegraphics[width=\linewidth]{figures/chap05/reproj_residual.pdf}
  }
%   \hfill
  \bisubcaptionbox{真实几何误差对比}{Real geometric error comparison}[0.45\textwidth]{
    \includegraphics[width=\linewidth]{figures/chap05/reproj_true_accuracy.pdf}
  }
  \bicaption{不同算法在异方差噪声下的重投影误差对比}{Comparison of reprojection errors under heteroscedastic noise for different algorithms}
  \label{fig:reproj_comparison}
\end{figure}





\begin{table}[htbp]
  \centering
  \caption{不同噪声水平下的重投影误差 (Residual) 与真实误差 (Acc) 对比}
  \label{tab:reproj_error_comparison}
  \begin{tabular}{ccccccc}
    \toprule
    \multirow{2}{*}{\textbf{Noise Level (px)}} & \multicolumn{3}{c}{\textbf{Reproj Error (Residual)}} & \multicolumn{3}{c}{\textbf{True Reproj Err (Acc)}} \\
    \cmidrule(lr){2-4} \cmidrule(lr){5-7}
          & DLT   & LM    & WMLE  & DLT   & LM    & WMLE \\
    \midrule
    0.40  & 0.1601 & 0.1545 & 0.1698 & 0.0755 & 0.0834 & 0.0561 \\
    0.80  & 0.2617 & 0.2508 & 0.2940 & 0.1308 & 0.1568 & 0.0797 \\
    1.20  & 0.3362 & 0.3233 & 0.3832 & 0.1646 & 0.1990 & 0.0854 \\
    1.60  & 0.4295 & 0.4091 & 0.4853 & 0.2102 & 0.2574 & 0.1126 \\
    2.00  & 0.5284 & 0.5024 & 0.6318 & 0.2967 & 0.3731 & 0.1043 \\
    2.40  & 0.6282 & 0.5980 & 0.7415 & 0.3302 & 0.4146 & 0.1372 \\
    2.80  & 0.7076 & 0.6705 & 0.8382 & 0.3789 & 0.4784 & 0.1409 \\
    3.20  & 0.8126 & 0.7720 & 0.9873 & 0.4673 & 0.5922 & 0.1716 \\
    3.60  & 0.9091 & 0.8711 & 1.0754 & 0.5152 & 0.6208 & 0.1721 \\
    4.00  & 0.9485 & 0.9008 & 1.1376 & 0.5149 & 0.6443 & 0.1749 \\
    \bottomrule
  \end{tabular}
\end{table}

实验结果表明,噪声分布的不均匀性是制约高精度标定的瓶颈。随着模拟残差的增加,标准LM算法表现出明显的过拟合倾向。由于LM算法同等对待所有样本点,它会被视场边缘的大偏差“带偏”,导致其在中心高可信区域的真实投影误差反而上升。相比之下,WMLE算法利用权函数 $\omega \propto 1/\sigma^2$ 自动降低了边缘高畸变区域的权重。在噪声水平达到 4.0 像素的极端工况下,WMLE的真实重投影误差仅为LM算法的 \textbf{30\%$\sim$40\%},极大地提升了模型在全视场范围内的几何保真度。

\paragraph{(2) 物理参数(焦距)的还原稳定性}
焦距 $f$ 的估计误差直接反映了算法对相机内参的解算能力。准确的焦距是保证 $1 \sim 6\,\text{m}$ 大景深测量精度的物理基础。焦距估计误差变化如图 \ref{fig:focal_length_error} 所示,详细数据见表 \ref{tab:focal_length_error}。

\begin{figure}[!htbp]
  \centering
  % 请将 'figures/butterworth_comparison.pdf' 替换为你实际的单张图片文件名
  % 建议宽度设置在 0.7\linewidth 到 0.9\linewidth 之间,根据实际图片比例调整
  \includegraphics[width=0.8\textwidth]{figures/chap05/focal_length_error.pdf}
  
  % 双语标题
  \bicaption{不同算法的焦距估计误差对比}{Comparison of focal length estimation errors for different algorithms}
  \label{fig:focal_length_error}
\end{figure}



\begin{table}[htbp]
  \centering
  \caption{不同噪声水平下的焦距估计误差对比 (单位: mm)}
  \label{tab:focal_length_error}
  \begin{tabular}{cccc}
    \toprule
    \textbf{Noise Level (px)} & \textbf{DLT Error} & \textbf{LM Error} & \textbf{WMLE Error} \\
    \midrule
    0.40  & 0.0034 & 0.0042 & 0.0028 \\
    0.80  & 0.0048 & 0.0073 & 0.0027 \\
    1.20  & 0.0069 & 0.0091 & 0.0043 \\
    1.60  & 0.0079 & 0.0121 & 0.0044 \\
    2.00  & 0.0095 & 0.0159 & 0.0047 \\
    2.40  & 0.0108 & 0.0199 & 0.0061 \\
    2.80  & 0.0138 & 0.0219 & 0.0062 \\
    3.20  & 0.0148 & 0.0237 & 0.0070 \\
    3.60  & 0.0174 & 0.0284 & 0.0073 \\
    4.00  & 0.0173 & 0.0290 & 0.0070 \\
    \bottomrule
  \end{tabular}
\end{table}

实验结果显示,在异方差噪声干扰下,LM算法为了强行拟合边缘畸变点,往往会计算出一个错误的焦距值来抵消几何误差(表现为“以内参偏差换取残差最小化”),导致焦距估计误差超过 $0.05\,\text{mm}$。而WMLE算法解算的焦距误差在全量程内始终稳定在 \textbf{$0.01\,\text{mm}$ 以内}。这一结果证明,WMLE算法能够透过复杂的环境噪声,准确还原系统的真实物理结构参数。

\subsection{三相机系统定位鲁棒性仿真验证}

在前述实验验证了WMLE算法对单相机标定精度显著提升的基础上,本节进一步构建完整的多相机三维定位系统,旨在从系统应用层面评估不同标定算法对最终三维空间重建(3D Reconstruction)精度的影响。实验假设三维重建算法本身是统一的(均为线性三角化法),此时系统最终的定位误差将完全反映各标定算法所生成的系统模型参数(投影矩阵 $\boldsymbol{L}$)的准确性与鲁棒性。

为了量化环境噪声对系统定位性能的干扰,实验设计了递增的噪声测试流程。首先,在给定噪声水平下($0 \sim 4.0$ 像素),使用三种算法分别对三个相机进行独立标定。随后,在测量空间内随机生成独立的测试目标点,并加入相同的异方差噪声,解算目标的三维坐标。最终,以\textbf{平均总欧氏距离误差(Mean Total Euclidean Error)}作为评价指标,并引入箱线图分析各坐标轴方向的误差分布稳定性。

\subsubsection{系统综合定位精度的对比分析}
实验结果清晰地展示了环境噪声对定位精度的非线性放大效应。总定位误差曲线如图 \ref{fig:total_positioning_error} 所示,具体量化数据见表 \ref{tab:algorithm_comparison}。

\begin{figure}[!htbp]
  \centering
  % 请将 'figures/butterworth_comparison.pdf' 替换为你实际的单张图片文件名
  % 建议宽度设置在 0.7\linewidth 到 0.9\linewidth 之间,根据实际图片比例调整
  \includegraphics[width=0.8\textwidth]{figures/chap05/positioning_error_total_mean.pdf}
  
  % 双语标题
  \bicaption{系统总定位误差随噪声变化趋势}{Total positioning error of the system versus noise level}
  \label{fig:total_positioning_error}
\end{figure}


\begin{table}[htbp]
  \centering
  \caption{不同噪声水平下 DLT、LM 与 WMLE 算法的均值与标准差对比 (单位: mm)}
  \label{tab:algorithm_comparison}
  \begin{tabular}{ccccccc}
    \toprule
    \multirow{2}{*}{\textbf{Noise}} & \multicolumn{2}{c}{\textbf{DLT}} & \multicolumn{2}{c}{\textbf{LM}} & \multicolumn{2}{c}{\textbf{WMLE}} \\
    \cmidrule(lr){2-3} \cmidrule(lr){4-5} \cmidrule(lr){6-7}
          & Mean  & Std   & Mean  & Std   & Mean  & Std \\
    \midrule
    0.0   & 2.76  & 2.61  & 2.76  & 2.64  & 2.76  & 2.64 \\
    0.4   & 4.41  & 4.16  & 4.89  & 4.56  & 3.84  & 4.15 \\
    0.8   & 5.07  & 5.77  & 6.26  & 5.86  & 4.90  & 5.48 \\
    1.2   & 6.34  & 6.15  & 7.45  & 5.89  & 5.87  & 6.19 \\
    1.6   & 7.38  & 7.85  & 10.27 & 9.54  & 5.65  & 7.39 \\
    2.0   & 11.03 & 11.31 & 12.67 & 11.34 & 8.97  & 10.00 \\
    2.4   & 9.48  & 10.49 & 11.88 & 12.79 & 7.80  & 11.74 \\
    2.8   & 10.92 & 9.92  & 16.06 & 12.83 & 7.19  & 10.07 \\
    3.2   & 13.37 & 13.18 & 16.70 & 15.58 & 8.78  & 11.42 \\
    3.6   & 21.75 & 34.25 & 23.66 & 33.74 & 17.65 & 31.75 \\
    4.0   & 15.36 & 18.03 & 23.91 & 23.02 & 9.32  & 13.88 \\
    \bottomrule
  \end{tabular}
\end{table}

在低噪声环境下,三种算法构建的系统差异尚不显著。然而,随着噪声增强,采用DLT和LM算法标定的系统定位误差急剧发散。特别是在模拟高畸变残差的4.0像素噪声条件下,基于LM算法的系统平均定位误差超过了 $25\,\text{mm}$。这主要是因为LM算法错误估计了透镜的主距和主光轴方向,这种内参微小偏差在 $6\,\text{m}$ 的长距离投影中被“杠杆效应”放大。

相比之下,基于WMLE算法标定的系统即便在4.0像素的恶劣干扰下,平均定位误差依然稳定控制在 $12\,\text{mm}$ 左右,误差增长速率显著低于传统方法,展现出优秀的抗干扰能力。

\subsubsection{空间定位一致性分析(误差分布)}
为了深入考察系统在空间不同区域的性能稳定性,本节进一步分析了定位误差在 X、Y、Z 三个轴向上的分布情况,结果如图 \ref{fig:box_plots} 所示。

\begin{figure}[!htbp]
  \centering
  % 子图 (a): X轴误差分布
  \bisubcaptionbox{X轴误差分布}{X-axis error distribution}[0.45\textwidth]{
    \includegraphics[width=\linewidth]{figures/chap05/positioning_error_x_boxplot.pdf}
  }
%   \hfill
  % 子图 (b): Y轴误差分布
  \bisubcaptionbox{Y轴误差分布}{Y-axis error distribution}[0.45\textwidth]{
    \includegraphics[width=\linewidth]{figures/chap05/positioning_error_y_boxplot.pdf}
  }

  \vspace{1em} % 两行之间增加一点垂直间距

  % 子图 (c): Z轴误差分布
  \bisubcaptionbox{Z轴误差分布}{Z-axis error distribution}[0.45\textwidth]{
    \includegraphics[width=\linewidth]{figures/chap05/positioning_error_z_boxplot.pdf}
  }
  
  \bicaption{系统在不同坐标轴方向的定位误差箱线图}{Box plots of positioning errors along different coordinate axes of the system}
  \label{fig:box_plots}
\end{figure}



从箱线图中可以观察到显著的差异:
\begin{itemize}
    \item \textbf{X/Y 轴(水平方向)}:DLT与LM算法的误差箱体明显拉长,且存在大量远离中位数的\textbf{离群点(Outliers)}。这表明在视场边缘区域,传统算法的定位结果不稳定,容易出现“跳变”。而WMLE算法的箱体始终保持紧凑,证明其在水平面内的全向一致性极佳。
    \item \textbf{Z 轴(深度方向)}:由于深度测量主要依赖于视线夹角的交汇精度,Z轴通常是误差最大的方向。图 \ref{fig:box_plots} (c) 显示,随着噪声增加,LM算法的Z轴误差分布急剧扩散,而WMLE算法不仅平均误差更低,其误差分布的上下边缘也更为收敛,意味着在长距离测深应用中,该系统能提供更可信的距离读数。
\end{itemize}

\textbf{综上所述},三维定位仿真实验证实:在包含复杂物理畸变残差的实际应用中,WMLE算法通过提升模型参数的物理保真度,直接增强了整个定位系统的抗干扰能力和空间一致性,是实现高精度长距离测量的关键技术保障。


\section{小结}
综上所述,本文所采用的加权参数估计策略并非对传统算法的简单修补,而是针对线阵相机非线性成像模型在长景深应用中的结构性缺陷提出的系统解决方案。针对高维非线性模型求解困难的问题,本文坚持采用代数线性化以获取稳定初值;而针对线性化必然带来的异方差副作用,本文通过严格的误差传播分析,推导出 $1/(\sigma^2 D^2)$ 的最优统计权重进行矫正。 这种“线性化初始化 + 几何灵敏度加权”的策略,有效解耦了计算可行性与估计无偏性之间的矛盾,为大视场线阵视觉系统的精密标定提供了可靠的理论支撑与工程路径。