% !Mode:: "TeX:UTF-8"
% \chapter{参考文献}
% \label{cha:bib}
% 参考文献可以直接写在 \texttt{thebibliography} 环境里, 利用 \cs{bibitem} 罗列文献条目.
% 虽然费点功夫, 但是好控制, 各种格式可以自己随意改写.

% 本模板推荐使用 BIB\TeX, 样式为 GB/T 7714—2015 《信息与文献-参考文献著录规则》, 是中国的参考文献推荐标准.
% 国内的绝大部分学术期刊、学位论文都使用了基于该标准的格式.

% 根据规则第10.2.2条, 正文中引用多著者文献时, 对欧美著者只需标注第一个著者的姓, 其后附“et al.”; 对于中国著
% 者应标注第一著者的姓名,其后附“等”字. 姓氏与“et al.”“等”之间留适当空隙.

% 尽管与学校提供的参考文献格式略有不同, 但更为正式.

% 看看这个例子: \citet{drl_2018} 总结了现状,\citet{narasimhan_language_2015} 提出了新结构, 
% 关于书的~\cite{tex1989,algebra2000}, 还有这些 \cite{narasimhan_language_2015,
% drl_2018,nikiforov2014,BuFanZhou2016:Z-eigenvalues,HuQiShao2013:Cored-Hypergraphs,
% KangNikiforov2014:Extremal-Problems,LinZhou2016:Distance-Spectral,
% LuMan2016:Small-Spectral-Radius,Nikiforov2017:Symmetric-Spectrum,Qi2014:H-Plus-Eigenvalues}.

% 如果需要使用上海大学的引用格式, 可以将模板类文件中
% \begin{verbatim}
% \RequirePackage[numbers,sort&compress,square,super]{natbib}
% \end{verbatim}
% 一行改为:
% \begin{verbatim}
% \RequirePackage[numbers,sort&compress]{natbib}
% \end{verbatim}

% 有时候一些参考文献没有纸质出处, 需要标注 URL.
% 缺省情况下, URL 不会在连字符处断行, 这可能使得用连字符代替空格的网址分行很难看.
% 如果需要, 可以将模板类文件中
% \begin{verbatim}
% \RequirePackage{hyperref}
% \end{verbatim}
% 一行改为:
% \begin{verbatim}
% \PassOptionsToPackage{hyphens}{url}
% \RequirePackage{hyperref}
% \end{verbatim}
% 使得连字符处可以断行. 更多设置可以参考 \texttt{url} 宏包文档.


\chapter{分布式单天线节点的CSI系统分析与数据预处理}
\label{chap:csi_system_design}


\section{引言}
作为协同系统中的射频感知部分,高质量的无线信号特征是构建高精度室内定位系统的基石。在基于信道状态信息(Channel State Information, CSI)的指纹定位或深度学习方案中,输入数据的纯净度与特征的可分性直接决定了定位模型的性能上限。然而,在实际部署的分布式感知网络中,受限于商用WiFi设备(如ESP32模组)的硬件能力,原始CSI数据往往伴随着高维冗余、环境噪声以及由非严格时间同步引起的相位畸变。如何从受污染的原始测量值中提取出能够鲁棒表征目标空间位置的物理特征,是实现协同定位前必须解决的首要难题。

针对上述挑战,本章将从物理层信号传播机理与数据层信号处理两个维度展开深入分析。首先,针对分布式单天线节点的硬件特性,本章基于多径效应理论,推导了移动目标对CSI信号的扰动机制,建立了幅值干涉模型,从理论上论证了在非同步系统中采用CSI幅值作为主要定位特征的合理性。其次,为了解决原始数据存在的“维数灾难”与噪声干扰问题,本章设计了一套完整的数据预处理流水线,涵盖了异常值剔除、基于巴特沃斯滤波的噪声抑制以及基于主成分分析(PCA)的特征降维。

本章的研究内容为后续构建基于深度神经网络的特征映射模型提供了物理可解释性依据与高质量的数据基础。


\section{分布式单天线节点的CSI幅值干涉模型}
\label{sec:csi_model}

CSI技术通过测量信道频率响应(Channel Frequency Response, CFR),能够细粒度地刻画室内无线信道的物理特性。在由分布式单天线节点(如本系统采用的ESP32模组)构成的感知网络中,信号的传播严格遵循多径效应(Multipath Effect)。为了阐明移动目标如何将其空间位移转化为CSI信号的可观测特征,并从物理层论证选用幅值特征而非相位特征的合理性,本节基于波的相干叠加原理建立了数学模型。

\subsection{信号叠加的数学表征}

在典型的室内多径环境中,接收端获取的信道状态信息 $H(f, t)$ 可视为静态路径分量与动态路径分量的矢量叠加。对于特定载波频率 $f$ 和时刻 $t$,其复数形式建模如下:
\begin{equation}
H(f, t) = H_s(f) + H_d(f, t) + N(t)
\end{equation}
其中 $N(t)$ 为加性高斯白噪声。为聚焦信号干涉机理,此处暂忽略噪声项的影响。信道响应 $H(f, t)$ 主要由两部分构成:

首先是\textbf{静态矢量} $H_s(f)$,它对应于视距传播(LoS)以及墙壁、静止家具等固定物体反射的信号。由于收发设备位置及环境背景相对固定,该分量的幅值 $A_s$ 和相位 $\phi_s$ 在短观测时间内保持恒定,即 $H_s(f) = A_s e^{j\phi_s}$。

其次是\textbf{动态矢量} $H_d(f, t)$,它由移动目标(散射体)的反射产生。设时刻 $t$ 目标的移动导致反射路径长度为 $L_d(t)$,则动态分量可表示为:
\begin{equation}
H_d(f, t) = A_d(t) e^{j\phi_d(t)}
\end{equation}
其中,$A_d(t)$ 为动态信号幅值。值得注意的是,由于反射损耗,动态幅值通常远小于静态幅值(即 $A_d \ll A_s$)。$\phi_d(t)$ 为动态相位,其变化与路径长度 $L_d(t)$ 呈线性关系:
\begin{equation}
\phi_d(t) = 2\pi f \tau_d(t) = \frac{2\pi}{\lambda} L_d(t)
\end{equation}
式中 $\lambda$ 为信号波长。该式表明,目标的微小位移($L_d(t)$ 的变化)将直接映射为动态相位 $\phi_d(t)$ 的旋转。

\subsection{幅值干涉特征的推导}

接收信号的功率(对应于 CSI 幅值的平方)取决于静态矢量与动态矢量的相干叠加。根据复数运算法则,总信号功率 $|H(f, t)|^2$ 可推导为:
\begin{equation}
\begin{aligned}
|H(f, t)|^2 &= |H_s + H_d(t)|^2 \\
&= (H_s + H_d(t))(H_s + H_d(t))^* \\
&= |H_s|^2 + |H_d(t)|^2 + 2|H_s||H_d(t)| \cos(\Delta \phi(t))
\end{aligned}
\end{equation}
其中,$\Delta \phi(t) = \phi_d(t) - \phi_s$ 表示动态路径与静态路径之间的相位差。

对上式各项的物理意义分析表明:$|H_s|^2$ 为直流分量(DC Component),代表稳定的环境背景能量;$|H_d(t)|^2$ 为二阶动态项,因数值极小通常可被忽略。决定信号波动特征的关键在于第三项——\textbf{交叉干涉项(Interference Term)}:
\begin{equation}
\label{eq:interference_term}
\text{Interference} \propto 2|H_s||H_d(t)| \cos\left(\frac{2\pi}{\lambda}L_d(t) - \phi_s\right)
\end{equation}
该干涉项表现出显著的“增益放大”效应:尽管动态反射信号 $H_d$ 本身能量微弱,但通过与强静态背景信号 $H_s$ 的耦合,目标由位置变化引起的相位旋转被转化为显著的幅值余弦震荡。因此,CSI 幅值的波动频率和包络特征,本质上是目标相对于接收节点几何位置关系的直接映射。

尽管理论上 CSI 的原始相位 $\angle H(f, t)$ 也包含目标位置信息,但在基于低成本硬件(如 ESP32)的分布式系统中,测量的相位值 $\hat{\phi}(t)$ 往往受到严重的非线性污染:
\begin{equation}
\hat{\phi}(t) = \phi_{\text{true}}(t) + 2\pi \Delta f_{\text{CFO}} t + \beta_{\text{SFO}} + Z_{\text{noise}}
\end{equation}
公式中,载波频率偏移($\Delta f_{\text{CFO}}$)会引入随时间累积的线性相位误差,而采样频率偏移($\beta_{\text{SFO}}$)及包检测延迟则会导致随机的相位跳变。在非连线同步的分布式网络中,消除这些时变误差极具挑战。相比之下,CSI 幅值特征 $|H(f, t)|$ 仅依赖于信号能量的标量叠加,天然免疫 CFO 和 SFO 产生的相位随机旋转。

综上所述,本研究采用\textbf{幅值干涉模型}作为理论基础,利用多径干涉现象将复杂的空间运动解耦为各分布节点上 CSI 幅值的特定扰动模式。这一策略在规避分布式系统严苛同步要求的同时,最大程度保留了包含目标位置信息的几何特征。然而,鉴于室内多径结构的复杂性,解析逆推位置极为困难,这为本文后续引入深度神经网络构建端到端的特征映射提供了充分的理论依据。


\section{CSI数据预处理与特征提取}
\label{sec:preprocessing}

在将CSI信号输入协同感知网络之前,必须对原始数据执行严格的预处理,以消除由商用硬件引入的噪声干扰并提取具有判别力的特征。本节详细阐述数据初始化、基于滤波的噪声抑制以及利用主成分分析(PCA)进行的特征降维与去冗余流程。

\subsection{数据初始化与幅度提取}

原始 CSI 数据从底层驱动获取后,表现为高维复数张量。设原始数据集合为 $H \in \mathbb{C}^{N_{\text{stream}} \times N_{\text{sub}} \times T}$,其中 $N_{\text{stream}}$、$N_{\text{sub}}$ 和 $T$ 分别表示空间流(Spatial Stream)数量、OFDM 子载波数量以及时间序列的采样长度。张量中的元素 $h_{i,j,k}$ 对应于第 $k$ 个空间流、第 $i$ 个子载波在第 $j$ 个采样时刻的复数信道响应。

鉴于在室内被动定位场景中,CSI 的幅度特征对人体运动引起的遮挡与反射效应更为敏感,且相位信息易受载波频率偏移(CFO)和采样时钟偏移(SFO)的非线性污染,本文主要聚焦于幅度特征的挖掘。首先,通过模运算提取幅度信息:
\begin{equation}
A_{i,j,k} = |h_{i,j,k}| = \sqrt{\Re(h_{i,j,k})^2 + \Im(h_{i,j,k})^2}
\end{equation}
后续的数据清洗及特征提取流程均基于该实数域的幅度张量 $A$ 展开。

\subsection{基于巴特沃斯滤波的噪声抑制}

商用 WiFi 设备采集的 CSI 幅度数据通常混杂着显著的高频环境噪声,这些噪声主要源于设备内部热噪声及环境电磁干扰。考虑到人体日常动作(如行走、起坐等)的运动频率主要集中在低频段(通常低于 40Hz),本文设计了一个 **6阶巴特沃斯(Butterworth)低通滤波器** 对幅度序列进行平滑处理。

该滤波器设定截止频率为 40Hz,旨在有效滤除频谱中的高频无关扰动,同时最大程度保留反映人体运动状态的低频动态成分。经滤波处理后,CSI 幅度曲线的信噪比显著提升,为后续特征提取提供了纯净的数据基础。


% 主要是做这个实验的目的是什么?
% 动-静止-动的过程:1.做了滤波和PCA后的CSI数据STFT 时频热力图与没做的对比图(没啥变化)2.取不取绝对值前后的 STFT 时频热力图变化大。

% 1.两个位置和静止时的CSI数据3D热力图对比、其相位图对比  







% \subsection{CSI数据的高维特性与冗余分析}

% 尽管去噪处理提升了信号质量,但原始 CSI 数据仍面临维度灾难与信息冗余的双重挑战。与粗粒度的接收信号强度(RSSI)不同,CSI 在频域上提供了细粒度的信道描述。

% 假设系统由 $T_{\text{node}}$ 个分布式感知节点组成,且每个节点包含 $p$ 个子载波(即 $N_{\text{sub}} = p$)。系统每一帧的观测数据构成了一个 $T_{\text{node}} \times p$ 的高维矩阵。直接利用该高维数据进行学习存在明显缺陷:首先,无线信道的相干带宽限制使得相邻子载波经历的多径衰落高度相关,导致频域特征存在大量冗余;其次,部分子载波可能处于频率选择性衰落的深衰落点,包含较多环境噪声。此外,过高的输入维度会呈指数级增加神经网络的参数量,极易引发模型过拟合。

% 针对上述频域强相关性(Frequency Correlation),本文引入主成分分析(PCA)技术,通过正交变换提取主要特征分量,将原始 $T_{\text{node}} \times p$ 维数据投影至低维子空间,实现去噪与降维的统一。

% \subsection{PCA降维算法流程实现}

% PCA 的核心思想是将原始数据投影到一组正交的新的坐标轴(主成分)上,使得投影后的数据方差最大化。针对本系统的分布式数据结构,我们将单帧 CSI 聚合数据定义为矩阵 $D \in \mathbb{R}^{T_{\text{node}} \times p}$:
% \begin{equation}
% D = \begin{bmatrix} 
% C_1^1 & C_1^2 & \dots & C_1^{p} \\ 
% C_2^1 & C_2^2 & \dots & C_2^{p} \\ 
% \vdots & \vdots & \ddots & \vdots \\ 
% C_{T_{\text{node}}}^1 & C_{T_{\text{node}}}^2 & \dots & C_{T_{\text{node}}}^{p} 
% \end{bmatrix}
% \end{equation}
% 其中,第 $i$ 行代表第 $i$ 个节点的全频域特征向量。降维过程的具体推导如下。

% 首先,为消除各子载波的直流分量偏移,需对数据进行**均值中心化**。计算数据矩阵 $D$ 每一列的均值,得到均值向量 $\mathbf{\gamma} \in \mathbb{R}^{1 \times p}$:
% \begin{equation}
% \mathbf{\gamma} = \left( \frac{1}{T_{\text{node}}} \sum_{i=1}^{T_{\text{node}}} C_i \right)^T
% \end{equation}
% 进而构建中心化矩阵 $U = D - \mathbf{1}_{T_{\text{node}}} \cdot \mathbf{\gamma}$,确保后续计算聚焦于数据的波动特性。

% 其次,为了量化子载波间的相关性,构建 $p \times p$ 维的**协方差矩阵** $V$:
% \begin{equation}
% V = \frac{1}{T_{\text{node}}} U^T U
% \end{equation}
% 矩阵 $V$ 中的元素 $V_{ij}$ 反映了不同子载波变化的协同程度。由于频域的高度冗余性,该矩阵往往是秩亏的。

% 随后,对 $V$ 进行**特征值分解**,求解方程 $V \mathbf{x} = \lambda \mathbf{x}$,得到 $p$ 个特征值 $\lambda_1 \ge \lambda_2 \ge \dots \ge \lambda_p$ 及其对应的单位特征向量。特征值的大小表征了对应主成分方向上的信号能量。通常,前几个较大的特征值对应于主要的多径结构变化(如人体运动),而微小的特征值则对应于随机测量噪声。

% 最后,依据累积方差贡献率(Cumulative Contribution Rate)选取前 $q$ 个最大的特征值对应的特征向量($q \ll p$),构建投影矩阵 $\Phi_{\text{pca}} = [\mathbf{x}_1, \dots, \mathbf{x}_q]$。利用该矩阵将中心化数据 $U$ 映射到低维空间:
% \begin{equation}
% D_{\text{pca}} = U \cdot \Phi_{\text{pca}}
% \end{equation}
% 变换后的特征矩阵 $D_{\text{pca}} \in \mathbb{R}^{T_{\text{node}} \times q}$ 有效剔除频域冗余并抑制了噪声子空间。该低维特征保留了对位置感知最敏感的主成分,将被重塑并作为深度神经网络的输入,从而显著提升模型的训练效率与定位鲁棒性。




% \section{分布式感知系统硬件架构与网络拓扑}
% \label{sec:hardware_architecture}

% 为了验证本文提出的协同室内定位算法,我们搭建了一套基于商用WiFi设备的分布式感知系统。该系统旨在通过低成本、易部署的单天线节点,构建覆盖目标区域的射频感知网络。

% \subsection{硬件选型与空间布局}

% 本系统选用 **ESP32-S3** 开发板作为基础感知节点。ESP32-S3 是一款集成 2.4GHz Wi-Fi 和蓝牙 5 (LE) 的高性能 SoC,其内置的底层驱动支持从物理层直接提取信道状态信息(CSI),且成本低廉,适合大规模分布式部署。

% 感知区域被定义为一个 $2.5\text{m} \times 2.5\text{m} \times 2.5\text{m}$ 的无遮挡(Line-of-Sight, LoS)长方体空间。为了实现全方位的多视角感知,系统共部署了 8 个 ESP32-S3 节点作为接收端(Station, STA),以及 1 个路由器作为发射端(Access Point, AP)。具体的空间布局如下:

% \begin{itemize}
%     \item \textbf{发射端(AP)}:放置于立方体空间底面的几何中心,负责持续发送全向 Wi-Fi 探测包。
%     \item \textbf{接收端(STA)}:8 个感知节点均匀分布在立方体空间的四个侧面外围。每个侧面布置 2 个节点,节点离地高度统一设置为 $0.7\text{m}$,同一侧面的两个节点间距为 $1\text{m}$。
% \end{itemize}

% 这种“中心激励-四周接收”的拓扑结构形成了密集的射频链路网络,确保了目标在区域内任何位置移动时,都能切割足够多的菲涅尔区(Fresnel Zone),从而引发可观测的信号波动。

% \begin{figure}[htbp]
%     \centering
%     % \includegraphics[width=0.8\textwidth]{figures/system_layout.png} 
%     % 请在此处插入你的系统布局示意图
%     \caption{分布式8节点感知系统空间布局示意图}
%     \label{fig:system_layout}
% \end{figure}

% \subsection{数据传输与时钟同步机制}

% 系统工作在 2.4GHz ISM 频段。为了捕捉细微的动作特征,接收端 CSI 的采样频率(Sampling Rate)设定为 **120Hz**。底层驱动基于 **ESP32-CSI-Tool** 框架开发,该工具能够解析 IEEE 802.11n 协议下的物理层前导码,提取包含幅度和相位的原始 CSI 复数矩阵。

% 在数据回传链路中,所有 ESP32-S3 节点与一台高性能 PC 主机处于同一局域网(LAN)内。节点将提取到的 CSI 数据包封装后,通过 **UDP 协议** 实时流式传输至 PC 服务器端进行记录与处理。

% 鉴于分布式系统各节点独立运行,为了实现多节点数据的时域对齐,本系统引入了 **SNTP(Simple Network Time Protocol)** 协议。
% \begin{enumerate}
%     \item 系统启动时,所有 ESP32-S3 节点首先通过网络连接至 NTP 服务器,同步世界协调时间(UTC)。
%     \item 在数据采集过程中,每个 CSI 数据包在产生瞬间均被打上微秒级的时间戳。
%     \item 服务器端依据时间戳对来自 8 个不同数据流的 CSI 帧进行重排序与对齐,消除了网络传输延迟带来的时序抖动。
% \end{enumerate}

