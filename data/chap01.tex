% !Mode:: "TeX:UTF-8"
\chapter{绪论}
\label{cha:intro} %这是 “绪论” 章节的标签,配合 \ref{cha:intro} 使用
 
\section{研究背景与意义}
党的二十大报告明确提出,要加快建设网络强国、数字中国,加快发展数字经济,推动物联网、大数据、人工智能等新一代信息技术与实体经济深度融合\cite{algebra2000}。作为数字经济的重要‘时空底座’,位置服务(Location Based Services, LBS)已成为支撑智慧城市建设、工业互联网发展以及社会精细化治理的关键核心技术之一[2]。特别是在国家大力推进‘新基建’的背景下,实现全场景、全天候的高精度位置感知,对于打通物理世界与数字世界的边界具有重大的战略意义。
目前,我国自主建设的北斗卫星导航系统已在室外开放空间取得了举世瞩目成就。然而,在卫星信号难以覆盖的室内复杂空间(如大型交通枢纽、地下管廊、智能工厂等),仍存在巨大的‘定位盲区’[3]。随着移动机器人的普及与人机协作需求的激增,室内定位技术已不再局限于简单的导航指引,而是上升为保障生产安全、提升物流效率及实现特定区域管控的核心环节。面对这一现状,如何突破室内复杂环境下的高精度定位瓶颈,构建自主可控且具备高鲁棒性的室内时空服务体系,已成为填补国家导航定位体系‘最后一公里’空白的紧迫需求[4]

随着室内位置服务(LBS)应用场景的不断拓展,学术界与工业界对定位系统的精度、鲁棒性及隐私安全性提出了更高的要求。在现有的主流技术中,基于面阵成像的视觉定位算法凭借其丰富的纹理特征提取能力,实现了较高的定位精度。然而,该类方法在实际部署中面临着突出的物理与伦理矛盾:一方面,在家庭、办公及保密单位等敏感区域,持续的面阵图像采集引发了用户对隐私泄露的严重顾虑,限制了其广泛应用;另一方面,成像视觉技术对环境光照高度敏感,在弱光、动态干扰或非视距(NLOS)遮挡条件下,其特征提取的稳定性显著下降,极易导致定位失效。尽管非成像式的线阵光电技术(如线阵CCD)因其高采样率和天然的隐私保护特性提供了新的硬件思路,但目前针对其在复杂空间定位中的研究仍相对匮乏。
与此同时,基于Wi-Fi的无线定位技术因无需额外布线而备受关注。虽然基于信道状态信息(CSI)的方法克服了传统接收信号强度(RSSI)粒度粗、易波动的缺陷,能够利用细粒度的频域特征有效刻画多径效应,但其高性能主要依赖于数据驱动的深度学习模型。这一特性带来了新的挑战:CSI指纹库的构建需要海量且带有精确位置标签的训练数据,而传统的人工定点采集方式不仅效率低下,且难以保证毫米级的标注精度。这种“数据饥渴”与高昂的人力成本,成为制约CSI高精度定位技术落地应用的关键瓶颈。
综上所述,单一的视觉模态受限于隐私风险与环境适应性,而单一的无线模态则受困于数据标注难题与信号漂移。面对复杂多变的室内环境,单纯依赖物理层面的单一传感器难以突破现有的性能天花板。因此,探索视觉(光)与射频(电)的异构协同机制,对于构建复杂场景下的全天候鲁棒定位具有重要的研究价值与应用前景。 


\section{国内外研究现状}
\subsection{室内定位技术国内外研究现状}
\subsubsection*{(1)无线网络室内定位技术}
随着物联网与移动计算的发展,室内定位技术在公共安全、资产追踪及智能环境中的应用需求激增。由于室内环境存在复杂的非视距(NLOS)传播和多径效应,单一技术往往难以满足所有场景需求。目前,研究主要集中在挖掘现有无线网络(如WiFi、蓝牙)的潜力,利用新兴技术(如UWB、毫米波)实现高精度定位,以及通过多源融合与人工智能算法提升系统的鲁棒性。

1. 基于 WiFi 的定位技术研究
WiFi 定位因其基础设施广泛普及而备受关注,研究热点正从传统的接收信号强度(RSS)向信道状态信息(CSI)和精细时间测量(FTM)转变,以克服信号波动带来的精度限制。
在指纹定位方面,深度学习被广泛引入以提取更深层的特征。Qin 等人[288]提出了一种名为 CCpos 的 WiFi 指纹室内定位系统,该系统基于卷积去噪自编码器和卷积神经网络(CDAE-CNN)模型,有效提取了指纹特征,提升了定位性能。

为了解决视距与非视距传播的干扰,Si 等人[180]提出了一种基于 WiFi FTM(Fine Time Measurement)的室内定位方法,该方法集成了 LOS/NLOS 识别机制,在复杂环境中提升了测距精度。此外,随着 WiFi 标准的演进,Storrer 等人[178]指出 IEEE 802.11ax 标准可提供高达 80 MHz 的带宽,对应的距离分辨率约为 1.88 米,这种高带宽特性为基于多天线被动雷达的室内人员追踪提供了硬件基础。

2. 基于蓝牙(BLE)的定位技术研究
低功耗蓝牙(BLE)技术因其低成本和低功耗特性在室内定位中占据重要地位。近年来的研究主要围绕蓝牙 5.1 标准引入的到达角(AoA)特性,以及与惯导系统的深度融合展开。
针对 AoA 定位,Zhao 和 Yang[77]实现了一种基于蓝牙 5.1 标准的 AoA 室内定位系统,利用多天线阵列产生的相位差计算到达角,在仓库资产定位场景中以低成本实现了**亚米级(sub-meter)**的定位精度。为了进一步提升角度估计的准确性,He 等人[188]提出了一种基于多天线阵列的 AoA 估计方法,实验表明,相比于传统的多信号分类(MUSIC)算法,该方法的平均角度误差小于 3.9°。

在算法融合方面,Kong 等人[111]提出了一种自适应反馈扩展卡尔曼滤波(AFEKF)算法,用于融合 BLE 和行人航位推算(PDR)。该机制将距离测量结果深度反馈到下一时刻的位置估计中,实验结果显示,与经典 EKF 算法相比,该算法将定位精度提高了 23.4\%。此外,Echizennya 和 Kondo[187]利用深度神经网络(DNN)处理 BLE 信标信号,实现了同时检测行人位置和运动方向,实验测得定位精度为 0.439 m,平均方向识别准确率达到 81.2\%。

3. 基于超宽带(UWB)与毫米波的高精度定位研究
对于对精度要求极高的工业和无人机场景,UWB 和毫米波(mmWave)技术因其极高的时间分辨率和空间扫描能力成为研究热点。

在 UWB 应用中,Queralta 等人[52]针对 GNSS 拒止环境下的无人机定位,提出了一种基于 UWB 的定位算法。该系统不仅能耗低,而且能够将定位误差控制在 0 到 4 cm 之间,且超过 50\% 的情况下总误差小于 3 cm。为了解决复杂环境下的非高斯噪声问题,Zhou 等人 [113]提出了一种自适应最大相关熵无迹卡尔曼滤波(AMCUKF)算法融合 IMU 和 UWB 数据,有效提升了系统的鲁棒性。

在毫米波定位方面,Jia 等人 [235]提出了一种改进的最小均方算法来优化 AoA 估计,并结合修正的多径 AoA-TOA 无迹卡尔曼滤波(UKF)算法。实验证明,该方法仅利用单个接入点(AP)即可在办公室环境中实现厘米级的定位精度,并获得了 2 倍的角度估计增益。

4. 基于多源融合与新兴技术的创新研究
为了应对单一技术的局限性,多源信息融合(如 PDR、地磁、视觉等)以及利用智能反射面(RIS)等新技术重塑信道环境,成为当前的重要研究趋势。
在多源融合滤波算法方面,Silva 等人[119]提出了 TrackInFactory 方案,利用紧耦合粒子滤波融合 INS 和 WiFi 信息,并通过一种新的可靠性度量动态更新粒子权重,在工业车辆追踪中实现了 0.81 m 的平均误差。Chen 等人[120]则提出了一种带有信息共享机制的联邦粒子滤波(FPF)算法融合 PDR 和 WiFi,通过主滤波器和子滤波器的协同,将定位误差控制在 1 m 左右。

此外,针对太赫兹(THz)频段的未来应用,Fan 等人 [243] 探索了深度学习在射频定位中的应用,提出了一种结构化的双向长短期记忆(Bi-LSTM)循环神经网络架构,实现了平均距离误差为 0.27 m 的 3D 室内定位。而在智能反射面(RIS)辅助定位方面,Zhang 等人[142] 的研究表明,通过优化 RIS 的反射波束赋形,可以将定位精度提升至分米级甚至厘米级,有效解决了信号遮挡问题。

\subsubsection*{(2)光学室内定位技术}
基于光学的定位系统在动作捕捉、虚拟现实(Virtual Reality, VR)、增强现实(Augmented Reality, AR)、机器人控制等领域有着广泛的应用,具有精度高、实时性强、延时低等优点。线阵相机、面阵相机、激光雷达等均是常见的光学传感器。
根据定位方式的不同,可将光学室内定位技术分为非自主定位和自主定位。

非自主光学定位技术是指在被追踪目标上附着特定的标记点(Fiducials),利用外部摄像机网络捕捉标记点的图像坐标,进而通过立体视觉原理重构目标六自由度(6DoF)位姿的技术。根据标记点成像机理的不同,目前的主流技术路线主要分为被动反光式(Passive)与主动发光式(Active)两大方向。
被动式技术通常采用涂有逆反射材料的球体或平面圆盘作为标记点,通过反射追踪相机发出的红外光进行成像。
英国Oxford Metrics Limited公司开发的VICON[29]是世界上第一款用于动画制作的光学动作捕捉系统,由一组高速红外相机、多个被动反光标记点和相应的软件组成。红外相 机被固定在房间四周,每台相机均朝向房间中心,确保相机镜头视野能够覆盖所有需要定位的区域。被动反光标记点被固定需要定位在目标物体上,当多个相 机同时观察到标记点时,该点的空间坐标即可被计算出来。

由于其具有无线缆束缚、重量轻且易于消毒等优势,长期以来一直是医疗导航领域的首选方案。文献 [1-12] 指出,商业化的光学追踪解决方案已广泛应用于微创手术中,其工具定位精度通常在 0.1 mm 量级。为了进一步提升精度,文献 [1-13] 展示了通过三目视觉(Trinocular)配置,可将被动系统的平移精度提升至 0.04 mm 左右 1。加拿大 NDI 公司开发的Polaris[33]系列产品采用双目红外 视觉方案,多应用于机器人辅助手术领域,被定位物体同样需要安装反光标记点。这些定位系统均可以达到亚毫米级定位精度,但是定位范围较小且成本较高。
然而,被动式标记点在复杂环境下的鲁棒性面临挑战。文献 [A] 的研究表明,被动标记点极易受到污染,导致反射率显著下降(如图 3 所示),从而严重干扰位姿追踪的稳定性 2。

中国度量科技开发 的Nokov[30]、美国Natural Point开发的Optitrack[31]以及瑞典的Qualisys[32]也属于非自主光学定位系统,其中Qualisys支持主动发光标记点,该功能可以给目标物 体添加专属身份标签。

主动式技术通过在目标上布置红外发光二极管(IR LEDs)主动发射信号,具有对环境光照不敏感、抗污染能力强等特点。
尽管主动式方案在鲁棒性上占优,但其引入的功耗与热效应不容忽视。文献 [1-21] 指出,为了维持足够的亮度,LED 往往需要较大的驱动电流,由此产生的热量不仅可能引起人体不适,甚至会影响标记点附近嵌入式半导体(如 FPGA 芯片)的性能稳定性 5。因此,如何在保证追踪精度的前提下优化驱动电流成为了研究重点。文献 [A] 的实验数据证明,在满足检测阈值的情况下,降低驱动电流并不会显著增加定位抖动,从而验证了低功耗驱动的可行性 6。

针对主动式与被动式系统在实际应用中的精度差异,文献 [B] 设计了一项包含两种测试环境(实验室与手术室)和两种工具长度的严格对比实验,评估了目标配准误差(TRE)。实验结果显示,主动式系统(Metronor OTS38/70)的整体精度(约 0.063 mm)显著优于被动式系统(Polaris Vega,约 0.259 mm),且在抗环境干扰方面表现出更低的标准差 7777。
除了传感器本身的差异,标记点的几何分布也是决定定位精度的核心因素。

自主光学定位系统强调定位主体的自主性,即移动终端(如智能手机、机器人)通过搭载的光电二极管(PD)或图像传感器(Camera/IS)接收环境中的可见光信号,在本地解算自身位置,无需依赖外部基站的测量。根据终端传感器的不同,目前的研究主要分为基于光电二极管的定位、基于图像传感器的定位以及多传感器融合定位。

1. 基于光电二极管(PD)与传感器辅助的定位研究
基于 PD 的系统通常利用接收信号强度(RSS)或到达角度(AOA)进行计算。为了克服单 PD 定位的局限性,研究者常引入惯性传感器辅助。

文献 [151] 提出了一种融合光强传感器与加速度计的定位方案,利用加速度计确定接收器方向以简化姿态计算。实验结果显示,在 $5\times5\times3$ m 的测试空间内,该方案的平均定位误差可控制在 25 cm 以内。针对办公走廊等典型场景,文献 [154] 利用智能手机内置的加速度计和磁力计感知自身姿态并获取法向量,实现了**亚米级(Sub-meter)**的定位精度。此外,文献 [155] 结合 RSS 与角度传感器(Angle Sensor),在 $1.2\times5\times2.5$ m 的实验空间中,将定位误差控制在 29.8~46.3 cm 之间。

2. 基于图像传感器(Camera)的视觉定位研究
随着智能终端摄像头的普及,基于图像传感器的 VLP 利用几何特征或卷帘快门效应实现高精度自主定位。

文献 [168] 将 VLP 应用于机器人操作系统(ROS),通过工业相机节点捕获 LED 图像并由定位器节点解算,实现了基于双灯定位原理的机器人自主导航。实验表明,该系统能提供 2 cm 以内的定位精度,且单次定位处理时间仅需 0.35 秒。为了进一步降低误差,文献 [169] 在移动机器人上应用了改进的校准程序和 SLAM 算法,在 $1\times1\times1.5$ m 的实验范围内,将定位误差降低至 0.82 cm。

针对 QR 二维码辅助定位,文献 [171] 提出了一种结合 VLP 与二维码的方案,接收器通过解码加载在 LED 灯上的二维码图像进行初始定位,在 $2\times2\times3$ m 的空间内实现了 4.0326 cm 的平均误差。

在算法对比方面,文献 [121] 提出的相机辅助 RSSR 算法(eCA-RSSR)与文献 [16] 提出的透视圆弧算法(V-PCA)进行了详细的数据对比。仿真数据显示:在接收器方向未知的情况下,V-PCA 算法能以 93\% 的概率将定位误差控制在 10 cm 左右,优于 eCA-RSSR 算法(78\% 的概率达到同等精度)7。此外,在覆盖率方面,当视场角(FoV)在 $25^{\circ}$ 至 $80^{\circ}$ 之间时,V-PCA 的覆盖率始终超过 90\%,比 eCA-RSSR 高出约 20\% 8。

3. 基于视觉/光电与惯性导航(INS/PDR)的融合定位研究
为了解决单一光学信号易受遮挡和视场限制的问题,融合 IMU 或行人航位推算(PDR)成为主流趋势,显著提升了系统的鲁棒性和精度。

文献 [136] 提出了一种基于扩展卡尔曼滤波(EKF)的视觉-惯性融合方法,利用卷帘快门相机和 IMU 共同完成定位。在 $5\times4\times2.3$ m 的实验场景中,该方法的均方根误差(RMSE)仅为 5 cm 9999。文献 [137] 利用单个 LED 的几何特征结合 IMU 估算接收器方向,在 $2.7\times1.8\times1.75$ m 的空间内实现了 5.44 cm 的定位精度 10101010。

针对行人导航,文献 [160] 利用粒子滤波(Particle Filter)同时处理 VLP 和 PDR 数据,在 $2.0\times2.0$ m 的测试区域内实现了 14 cm 的定位误差 1111111111111111。文献 [161] 则采用扩展卡尔曼滤波融合 RSS 和 PDR 数据,在 $2.2\times2.2$ m 的区域内取得了 14.5 cm 的定位精度 12121212。此外,文献 [140] 将智能手机 IMU 传感器与 PDR 及 VLP-AoA 算法结合,实验结果表明该方法能将平均误差降低至 0.85 m,实现了无缝移动室内导航 13131313。对于大场景应用,文献 [144] 在博物馆场景中利用 5G 无线电光定位系统,实现了 0.18 m 的平均定位误差 14。

值得注意的是,无论是主动还是被动系统,都严格依赖于视线(Line-of-Sight),一旦标记点被遮挡即会导致追踪中断。


\subsection{基于信道状态信息的室内定位技术国内外研究现状}
近年来,随着无线通信技术的发展,基于信道状态信息(Channel State Information, CSI)的室内定位技术因其比接收信号强度(RSSI)更细粒度的信道描述能力而备受关注。CSI能够捕捉多径效应下的幅度与相位信息,研究者们通过引入深度学习、迁移学习以及针对5G信号特性的新方法,显著提升了定位的精度与鲁棒性。

特征工程与鲁棒性优化 针对CSI信号的波动性,研究者通过精细化的特征处理来提升性能。[11]Che 等人 提出了FuFi方法,利用MIMO系统中全维度子载波的归一化幅度作为指纹,并配合改进的加权K近邻(IWKNN)算法,在会议室场景下实现了 0.65 m 的平均距离误差(MDE),相比传统FIFS算法精度提升了 59.9\%。[12]Sun 等人 结合PCA降维和SVM分类,并考虑了人体不同姿态(如站立、蹲下等)对CSI的影响,实验证明该策略使定位精度提升了 9.6%,平均误差降至 0.61 m。[13]Reyes 等人 探索了语音处理中的d-vector和i-vector嵌入技术,发现通过i-vector进行模型自适应调整,可将针对特定位置的检测准确率从 75.47% 提升至 80.62%。[14]Choi 则提出了一种传感器辅助的无监督学习技术,利用PDR(行人航位推算)轨迹来训练Wi-Fi测距模块,在没有人工标注数据的情况下,基于CNN的测距辅助定位实现了 1.038 m 的MAE,优于传统路径损耗模型的 1.356 m。

[15]Astafiev 等人 探索了仅利用 CSI 相位信息进行设备定位的可行性。针对相位数据的非线性特征,作者构建了一个全连接神经网络,利用 56 个子载波和 3 对天线的相位矩阵作为输入。实验结果显示,在静态场景下,该算法的分类定位准确率最高可达 94.58\%(使用 512 个神经元时)。然而,研究也指出,仅依赖相位信息在动态(移动)目标的定位上存在局限性,会产生较大的误差,因此建议未来结合幅度信息以提升动态场景下的鲁棒性。

深度学习模型的架构创新 深度神经网络(DNN)及其变体在处理复杂的CSI特征映射方面表现出卓越性能。针对传统方法在复杂环境下的局限性,[1]Li 等人 提出了一种基于Transformer的定位方法LoT(Localization Transformer),通过趋势周期分解填充和矩形Patching机制解决了CSI矩阵的长条形输入问题,实验表明该方法在室内场景下的平均绝对误差(MAE)仅为 0.18 m,优于ResNet等基准模型。[2]Song 等人 设计了双通道卷积神经网络系统DuLoc,利用孤立森林去除异常值并通过Mean Shift聚类划分指纹库,分别处理稳定与不稳定子载波,其在复杂室内环境下的平均距离误差降低至 0.42 m,且 85% 的测试样本误差控制在 0.5 m 以内。[3]Wan 等人 针对MIMO系统提出了ACPNet,引入自注意力(Self-Attention)和通道注意力(Channel Attention)机制来关注高质量的信道响应,相比于现有的SOTA模型(如PirnatEco),其均方根误差(RMSE)和平均距离误差(MDE)分别提升了 22.0% 和 41.1%。此外,[4]Astafiev 等人 提出了一种基于全连接神经网络的方法,利用56个子载波和9对天线的幅度数据,在静态场景下实现了高达 99.8% 的分类定位精度(误差容限0.5m)。

样本效率与半监督/迁移学习 为了降低指纹库构建的人力成本并解决数据孤岛问题,小样本学习和联邦学习成为研究热点。[5]He 等人 提出了基于半监督阶梯网络的方法LadderNetFi,该方法结合了去噪自动编码器(DAE),仅需 $10\%$ 的有标签数据即可达到与全监督学习相当的性能,在实验室场景下实现了 1.65 m 的RMSE。[6]Guo 等人 提出了联邦迁移学习框架FedPos,通过聚合不同用户模型的非分类层参数构建云端特征提取器,实验显示该方法在保护隐私的同时,相比普通训练方法平均定位性能提升了 5.22\%,且训练时间减少了约 34.78\%。[7]Zhang 等人 引入了基于改进TrAdaBoost的迁移学习系统,利用One-Hot编码和One-vs-Rest算法处理多分类问题,不仅使定位精度在动态环境下提升了 35%,还将现场勘测开销(SSO)降低了 40%。


面向5G新空口(NR)的CSI定位 随着5G商用部署,利用5G信号特性的CSI定位技术开始兴起。针对室内单基站场景下的定位难题,[8]Zhou 等人 提出了利用5G下行链路多波束特性的指纹定位方法,相较于单波束,多波束技术将WKNN、SVM和RF算法的$95\%$定位误差分别降低了 $99.7\%$、66.9\% 和 95.3\%。在另一项工作中,[9]Zhou 等人 结合ELM降维和CatBoost算法提出了EPL-CatLoc方法,在单gNB覆盖的办公室和走廊场景中,定位准确率分别达到了 $98.24\%$ 和 $94.10\%$。[10]Ruan 等人 开发了iPos-5G系统,利用无监督深度自编码器重构特征并结合改进的径向基函数(RBF),在办公室场景下实现了 2.14 m 的平均绝对误差(MAE),优于DeepFi等传统方法。[17]Cerar 等人 基于 CTW 2019 大赛的公开数据集,研究了利用卷积神经网络(CNN)处理单基站大规模 MIMO CSI 数据的潜力,在训练集与测试集随机分布的场景下,该结构相比于文献中现有的 CNN 和全连接网络模型,定位精度提升了 2 cm 到 10 cm,且所需的模型权重参数更少,计算效率更高。这证明了在处理高维 Massive MIMO 数据时,深层残差网络具有更强的特征提取能力。


\subsection{基于线阵CCD的室内定位技术国内外研究现状}
线阵CCD是一种光电转换器件,可将接收到的光信号转换成电荷并不断地累积起来,并以电压的形式输出,按照规定时序通过模数转换器(Analog-toDigital Converter, ADC)依次读取这些电压信号便可得到图像,具有单行光敏像 元多、像元尺寸相对灵活、感光范围大、扫描速度快等优点。

尽管目前基于面阵相机(Area Array Camera)的视觉SLAM技术已成为室内定位的主流,但其在千赫兹级(kHz)高速动态捕捉及嵌入式微处理器低算力场景下仍面临巨大挑战。相比之下,线阵CCD凭借其一维成像带来的数据压缩优势及微秒级响应速度,为解决上述瓶颈提供了极具潜力的差异化路径。

1.1 系统构型与成像机理的演变
早期的线阵CCD定位系统多采用多相机交会测量原理。[1]Macellari提出的CoSTEL系统是该领域的开创性工作,其利用三个一维传感器配合环形透镜,实现了对人体运动三维坐标的立体监测。在此基础上,[2]秦志军等针对大型射电望远镜(FAST)馈源模型的检测,设计了由三个线阵CCD和复合柱面透镜组成的光电定位系统,通过大视场透镜设计解决了大范围运动目标的实时跟踪问题。针对多目标测量中的干扰问题,[3]艾莉莉等设计了一种基于分光棱镜和滤光片的光学架构,构建了双点目标三维坐标重构子系统,有效解决了多相机与多合作目标一一对应时的信号串扰问题。

 [6]刘海庆等提出了一种正交柱面成像相机模型,在一个相机内集成两个正交放置的线阵CCD,分别测量目标的水平角和垂直角,配合双相机交会即可实现高精度三维测量。[5]杨凌辉等进一步发展了该技术,利用正交柱面成像相机结合光立体靶,仅需单目即可通过空间后方交会原理完成三维坐标测量,有效解决了大型装备测量中的视场遮挡难题。此外,[4]王闯等提出了一种反向定位思路,将由三个线阵CCD和柱面透镜组成的传感器安装在移动端,通过观测天花板上的红外LED锚节点进行自主定位,避免了移动轨迹的暴露,便于室内大范围扩展。[7]贺玉泉等则利用空间直线在成像面上的“积聚”特性,设计了双目线阵CCD平面定位系统,通过直方图均衡化增强图像对比度,实现了对目标平面的高精度定位。

1.2 相机标定技术
标定是保证线阵CCD测量精度的核心环节,主要包括几何参数标定和畸变校正。由于线阵CCD通常配合柱面镜或广角镜头使用,光学畸变不可忽视。[9]骆文博等建立了包含成像误差的一维摄像机修正模型,提出了“先畸变矫正、后参数标定”的两步法,显著提升了直接线性变换(DLT)的解算精度。[10]王国辉等针对立体线阵相机,提出了一种利用简单磁性棒靶标的两步标定法,先通过线性模型求解初值,再利用Levenberg-Marquardt算法进行非线性优化,有效解决了镜头畸变参数的求解问题。
为了简化标定流程,[8]赵霞等提出了一种内外参数分离的标定方法,即内参在实验室利用平行光管标定,外参在现场利用三叉模板标定,降低了现场操作复杂度。针对传统物理模型参数繁多且非线性强的问题,[11]李晶等引入了神经网络进行隐式标定,通过训练建立像点坐标与空间三维坐标的直接映射关系,避开了复杂的几何建模过程,实验证明该方法在坐标恢复精度上优于传统DLT方法。

1.3 光斑定位与信号处理
在室内复杂光照环境下,如何从背景噪声中精确提取合作目标(如LED)的亚像素质心是提升系统分辨率的关键。[12]尚学军等对比了重心法和质心法,提出了基于线性插值和权值非线性的改进质心算法,在不同噪声水平下均能保持较高精度,定位精度可达1/25像素。[13]曹苗等针对激光三角法测量中目标表面粗糙引起的散斑噪声,提出了基于区域划分和线性插值的改进灰度质心算法,通过并行下移阈值有效抑制了背景干扰。

除了算法改进,通过硬件与系统设计抑制背景光也是重要手段。[14]Ren和Kumar在LOSA-X系统中引入了背景光消除机制,通过控制标记点LED的开关,利用差分算法去除动态背景照明,使系统具备了在复杂光照甚至户外环境下工作的能力。 [15]李晶等对线阵CCD探测能力的理论分析表明,背景光照度、目标亮度及积分时间直接影响系统的极限探测距离和速度,合理的参数匹配是保证微弱目标被捕获的前提。


在位姿解算算法方面,[16]王艳等针对三线阵CCD系统,提出了一种改进的正交迭代算法(OI),通过建立新的物空间共线性误差目标函数并进行全局优化,解决了传统算法易陷入局部极值的问题,显著提高了位姿解算的收敛速度和精度。

为了进一步提升系统的动态性能和鲁棒性,[17]Kumar和Ren开发了空间目标跟踪系统,将线阵光学传感器与9轴惯性测量单元(IMU)进行融合。[18]在其后续的LOSA-X系统中,通过扩展卡尔曼滤波(EKF)算法融合视觉数据与惯性数据,不仅解决了视觉遮挡期间的数据丢失问题,还将位置追踪误差降低了55\%,实现了毫米级的定位精度和极高的更新率。与之相比,尽管商用系统如Optotrak能够提供极高的精度(微米级)[18],但其高昂的成本限制了大规模应用,而基于线阵CCD的融合方案在保证精度的同时大幅降低了成本。



% \begin{itemize}
% \item{main.tex}: 主文件, 包含封面部分和基本设置.
% \item{data}: 包含本文正文中的所有章节.
% \begin{itemize}
% \item{abstract.tex}: 中英文摘要.




\section{论文的研究内容和创新点}
\label{sec:first}
\subsection{论文的主要研究内容}



针对现有室内定位技术在数据获取人力时间成本、环境适应性及定位精度等方面的局限,本文提出了一种基于线阵视觉与信道状态信息(CSI)的协同定位架构。全文按照“理论建模—关键技术研究—系统协同验证”的逻辑展开,主要研究内容按章节安排如下:



\begin{itemize}
    \item \textbf{第一章:绪论} 
    阐述了室内定位技术的研究背景与意义,分析了基于视觉和 Wi-Fi CSI 定位技术的国内外研究现状。分析了现有单一模态定位面临的数据依赖度高与鲁棒性不足等问题,提出了将线阵视觉的高精度特性与 CSI 的广覆盖特性相结合的研究思路,并确立了本文的研究目标与技术路线。

    \item \textbf{第二章:分布式节点的信道状态信息模型分析与数据预处理} 
    实现了基于 ESP32 分布式节点的 CSI 特征提取与位置回归。从物理层面推导了单天线节点的 CSI 幅值干涉模型,分析了利用幅值特征表征空间位移的可行性。

    \item \textbf{第三章:基于信道状态信息的室内定位算法与系统设计} 
    设计了包含多尺度时序卷积(MSTC)、ConvNeXt 主干网络及坐标注意力机制(CoordAtt)的深度神经网络,并使用了结合位置误差与物理平滑约束的复合损失函数,以实现端到端的特征映射。
    针对定位结果的时域平滑与抗差跟踪问题进行了研究。考虑到室内复杂环境下观测噪声具有非平稳特性,构建了基于新息(Innovation)统计特性的自适应贝叶斯跟踪框架。推导了观测噪声协方差矩阵的极大似然估计公式,并将其应用于粒子滤波算法中,设计了融合动态噪声感知的自适应粒子滤波。该算法旨在根据新息序列实时调整似然函数形态,以改善粒子贫化现象,提升系统在非视距场景下的鲁棒性。    

    \item \textbf{第四章:线阵相机三维定位的成像模型与几何基础} 
    详细分析了线阵 CCD 配合柱面透镜的“一维针孔”成像机理,建立了包含世界坐标系、相机坐标系及像素坐标系的多层几何模型。推导了基于“Y型”布局的多线阵相机光平面交汇几何模型及其线性代数求解方法,为获取室内空间位置信息提供了理论支撑。

    \item \textbf{第五章:线阵传感器的参数标定方法} 
    针对视觉子系统的参数标定问题进行了深入研究。考虑到柱面透镜引入的非线性畸变,提出了一种分阶段标定策略:首先建立偶次多项式模型对镜头畸变进行校正,随后基于几何灵敏度与测量不确定性分析,提出了一种加权最大似然(WMLE)标定算法。该方法通过引入交叉相乘残差构建目标函数,并推导了基于像素噪声统计的权重计算方法,旨在提升非理想光学条件下的标定精度与数值稳定性。

    \item \textbf{第六章:协同系统结构和实验结果分析} 
    完成了光电协同系统的集成与全流程验证。提出了“离线监督-在线互补”的分层协同架构:在数据层,利用线阵视觉系统辅助构建 CSI 指纹库;在策略层,构建了基于视距状态感知的“主备切换”机制。搭建了实验平台,分别验证了线阵视觉标定算法、CSI 定位网络及自适应跟踪算法的性能,并对协同架构在解决室内定位相关问题上的有效性进行了实验分析。
\end{itemize}


\begin{figure}[htbp]
    \centering
    % 假设你的图片放在当前目录下的 figures 文件夹中,文件名为 framework.pdf
    \includegraphics[width=0.9\textwidth]{figures/1.pdf} 
    \caption{基于CSI与线阵视觉协同的室内定位系统总体架构图}
    \label{fig:framework}
\end{figure}

\subsection{论文的创新点}

\begin{enumerate}
    \item \textbf{线阵相机高精度三维定位原理与标定方法研究} \\
   针对室内高精度真值获取难题,设计了基于“Y型”布局的多线阵CCD视觉捕捉系统。建立了基于柱面透镜的一维针孔成像数学模型,推导了多相机光平面交汇的三维重构算法。重点研究了线阵相机的标定技术,提出了一种基于几何灵敏度与测量不确定性的加权最大似然(WMLE)标定方法,并通过“去畸变-线性解算”的分步策略,有效解决了大视场柱面镜头的非线性畸变与参数求解问题。

    \item \textbf{基于CSI幅值干涉模型的无源定位深度学习网络设计} \\
    构建了基于分布式ESP32-S3节点的低成本CSI感知网络。从物理层面推导了CSI幅值干涉模型,论证了利用幅值特征表征空间位移的可行性。针对CSI信号的时变性与多径效应,设计了一种融合多尺度时序卷积(MSTC)、ConvNeXt主干网络及坐标注意力机制(CoordAtt)的深度神经网络。该网络通过提取CSI的时空频多维特征,并结合加权移动平均(WMA)与物理约束损失函数,实现了高精度的端到端位置回归。

    \item \textbf{基于新息自适应的贝叶斯目标跟踪算法研究} \\
    为了解决室内复杂环境下观测噪声非平稳导致的滤波发散问题,构建了基于新息(Innovation)统计特性的自适应贝叶斯跟踪框架。推导了观测噪声协方差矩阵的极大似然估计公式,提出了基于遗忘因子的动态噪声更新策略。将该机制应用于卡尔曼滤波与粒子滤波中,设计了融合动态噪声感知的自适应粒子滤波算法,通过动态调整似然函数形态,解决了粒子贫化问题,提升了系统在非视距与强干扰场景下的鲁棒性。

    \item \textbf{“离线监督-在线互补”的光电协同架构实现与验证} \\
    提出了视觉与射频异构模态的深层协同机制。在数据层,利用SNTP协议实现了光电信号的微秒级同步,利用线阵视觉系统实现了CSI指纹库的自动化、高精度构建,突破了人工标注的效率瓶颈;在策略层,构建了基于视距状态感知的“主备切换”机制,利用视觉的高精度与射频的抗遮挡特性实现优势互补。搭建了实验平台,对各子系统性能及协同架构的有效性进行了系统性的实验验证。
\end{enumerate}










